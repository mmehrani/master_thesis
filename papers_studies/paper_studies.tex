\documentclass[12pt,onecolumn,a4paper]{article}
\usepackage{epsfig,amsthm,amsmath,booktabs,csquotes}
\usepackage [pagebackref=true, colorlinks, linkcolor=blue, citecolor=magenta, urlcolor=cyan] {hyperref}
\usepackage{color,xcolor}

\usepackage{subcaption}
\usepackage[labelformat=parens,labelsep=quad, skip=3pt]{caption}
\usepackage{graphicx}
\usepackage{enumerate,braket}
\usepackage[localise]{xepersian}
%\settextfont[Scale=1.2]{‌BNAZANIN.TTF}
\settextfont[Scale=1.2]{BZAR.TTF}
\setlatintextfont[Scale=1]{Times New Roman}





\begin{document}
\title{سیر مطالعاتی من برای ارائه پایان نامه کارشناسی ارشد} 
\author{محسن مهرانی - استاد راهنما: دکتر سامان مقیمی عراقی}
\date{}
\maketitle
\قسمت{مطالعه مقاله شماره \cite{PhysRevLett.105.158104}:} 
در این مقاله مدلی را مشاهده کردیم که به کمک مدل $KM$ یک شبکه نورونی کامل را توصیف کرده است. این شبکه شامل نورون‌های مهاری است که روشن شدن هر کدوم از آن‌ها باعث مهار شدن نورون‌های همسایه می‌شود. معادله تحول اختلاف پتانسیل هر کدام از نورون‌ها با محیط بیرونش از رابطه زیر داده می‌شود ($g:$ ضریب اتصال هر جفت نورون، $S:$ ماتریس اتصال، $t_d$ زمان تاخیر میان زدن تیزه و تحریک آن، $a_i$ یک پتانسیل تحریکی و خارجی):
\begin{align}
\dot{v_i}=a_i - v_i - \frac{g}{N} \sum_{n|t_n<t} S_{i,l(n)} \delta(t - t_n - t_d) 
\label{eq:potential_1}
\end{align}

پارامتر نظم سیستم را به کمک میدان ($E$) تعریف کرده است اما پارامتر نظم را انحراف از معیار آن در طول زمان معرفی کرده است.
\begin{align}
\ddot{E}+ 2\alpha \dot{E}+\alpha^{2}E &=2\alpha N \sum_{n|tـn<t} \delta(t - t_n - t_d) \\
\sigma^{2} &= \braket{E^{2}}_{t} - \braket{E}^{2}_{t}
\end{align}
در طول زمان میدان $E$ و $\sigma$ را رصد کرده است و دیده‌است که میدان خاموش و روشن می‌شود و انحراف از معیار آن مقدار خوبی مثبت است چنان که این خاموش و روشن‌ها را با معنا نشان می‌دهد. حال ادعای این مقاله است که این خاموش و روشن شدن‌ها الگویی آشوبناک دارند و ادعا کرده است که به اندازه متناهی سامانه نیز وابسته نیست.\\

\subsection{سوالات}

\begin{enumerate}
\item
مدل $Kuramoto$ به قرار زیر است. چطور معادله \ref{eq:potential_1} به آن تبدیل می‌شود. دلتای یاد شده در معادله \ref{eq:potential_1} دلتای دیراک است؟ یا دلتایی که بیشینه آن عدد یک است؟ 
\begin{align}
\frac {d\theta _{i}}{dt}=\omega _{i}+\sum _{j=1}^{N}a_{ij}\sin(\theta _{j}-\theta _{i}),\qquad i=1\ldots N
\end{align}
\item
$t_n$ چیست؟
\item
اگر قرار باشد جمعی که در رابطه \ref{eq:potential_1} نوشته‌ایم روی تمام زمان‌های از ازل تا $t$ باشد پس آیا هر نورون حافظه‌ای از کل رخدادهای گذشته دارد؟ حتی از لحظاتی که قبل از تیزه زدن ها وجود دارند؟
\item
میدان $E$ به چه معناست؟ چطور تعریف کردیم؟ آیا مشخصه‌ای از کل سیستم است؟
\end{enumerate}

\textbf{پاسخ استاد:}
\begin{enumerate}
\item
قرار نیست کوراموتو به این تبدیل بشود. ممکنه یه شباهت‌های کلی‌ (به این معنی که مثلا دور می‌زنند) باشه ولی کلا دو تا معادله‌ی متفاوتند. در ضمن تابع دلتای دیراک است.
\item
کمیت‌های $t_n$ زمان‌هایی است که تیزه‌ای در سیستم زده می‌شود. [*می‌گویم: پس احتمالا معادله دیفرانسیلی ما دائم در حال به روز کردن سمت راست خودش است. هر وقت نورونی تیزه زد آن را در جمله سمت راست ذخیره می‌کنیم. پس احتمالا تقارن زمانی نداریم مگر پس مدتی طولانی که تاثیر شرایط اولیه بسیار کوچک دیده شود.]
\item
داستان اینه که هر نورونی که تیزه بزنه، اطرافیانش رو تحت تاثیر قرار می‌ده. پس وضعیت نورون به تمام تیزه‌های زمان‌های قبل وابسته است.
\item
هر وقت در هر جای دستگاه، تیزه‌ای زده بشه، کمیت$E$ کمی بالا می‌ره و بعد افت پیدا می‌کنه. حالا اگر تند و تند جا‌های مختلف تیزه زده بشه، این کمیت کم و بیش مقداری غیر صفر پیدا می‌کنه.[این کمیت را خودمون تعریف کرده‌ایم که بر حسب پارامترهای سیستم متحول می‌شود. مانند یک آشکارساز که به سامانه متصل می‌شود تا اندازه‌گیری خود را با یک عقربه نشان دهد.] اما اگر این تیزه زدن‌ها همگام باشه، یعنی همه با هم یه زمانی بزنند و بعد یه مدتی خاموش باشند، این کمیت، اول کلی زیاد می‌شه و بعد یه مدتی کم می‌مونه و در نتیجه انحراف معیارش زیاد می‌شه.
\end{enumerate}

\subsection{شبیه سازی مدل پیاده شده در مقاله}
یکی از مشکلات شبیه سازی معادلات دیفرانسیلی حضور تابع دلتای دیراک است. این تابع در نقطه صفر خود دارای مقداری بینهایت است. برای برطرف کردن این معذل چه باید کرد؟ نکته در این جا نهفته است که چون ما برای حل عددی معادله دیفرانسیلی خود از زمان پیوسته استفاده نمی‌کنیم و از گام‌هایی با طول مثبت $\Delta t$ استفاده می‌کنیم این مشکل به صورت زیر مدیریت می‌شود.
\begin{align}
v_{i}(t+\Delta t) &= v_{i}(t) + \int_{t}^{t+\Delta t} \dot{v_i}  dt \\
&= v_{i}(t) + \int_{t}^{t+\Delta t} \left[ a_i - v_i - \frac{g}{N} \sum_{n|t_n<t} S_{i,l(n)} \delta(t - t_n - t_d)  \right]   dt \\
&\approx v_{i}(t) +  \left[ a_i - v_i(t) \right] \Delta t - \frac{g}{N} \sum_{n|t_n<t} S_{i,l(n)} \int_{t}^{t+\Delta t} \delta(t - t_n - t_d) dt  \\
&\approx v_{i}(t) +  \left[ a_i - v_i(t) \right] \Delta t - \frac{g}{N} \sum_{n|t_n<t} S_{i,l(n)} H(t + \Delta t- t_n - t_d) \label{eq:potential_changes}
\end{align}

حالا تابع پله کاملا برای ما آشنا و قابل مدلسازی است. دقت شود که تابع پله یاد شده فقط در محدوده $t, t+\Delta t$ زندگی می‌کند و پس از آن اعتبار ندارد. معادله \ref{eq:potential_changes}  می‌گوید که باید برای تحول پتانسیل نورون $i$ام بررسی کنیم که آیا نورونی در همسایگی آن تیزه زده است یا نه. اگر چنان باشد یک واحد به جمع تیزه زدگان اضافه کنیم.
%\newpage
\bibliographystyle{plain-fa}
\bibliography{MyReferences}

\end{document}


