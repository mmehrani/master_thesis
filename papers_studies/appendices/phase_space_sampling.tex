\فصل{نمونه‌گیری صفحه فاز}
\section{انباشت‌وشلیک} \label{appendix:phase_sampling_if}
شکل \ref{fig:if_g_d_phase_space} نمایشگر صفحه‌ی فاز است که مقادیر متفاوتی را برای هر نقطه گزارش می‌کند. بدون شک داده‌گیری از تک‌تک نقاط فضا امکان پذیر نبود و برای ایجاد این شکل یک شبکه از نقاط به فاصله‌ی نزدیک از هم استفاده شد و رنگ نقاط میانی از نقطه‌ی همسایه‌ی نزدیک آن‌ها برداشته شد.
\begin{figure}[h]
	\includegraphics{../scripts/all_neurons_model_in_one_place/IF_ensembles/N10000_T1000_I1.2_2.8_cluster_computed/sigma_phase_space_contour_points_plotted.png}
	\caption{نقاطی از صفحه فاز که از آن‌ها داده‌گیری انجام شده است به رنگ سفید درآمده‌اند.}
	\label{fig:if_g_d_phase_if_space_points_plotted}
\end{figure}

\section{چرخنده} \label{appendix:phase_samplingـrotational}
\begin{figure}[h]
	\includegraphics{../scripts/all_neurons_model_in_one_place/Rotational_ensembles/N10000_T100_I9.5_13.5_cluster_computed/sigma_phase_space_contour_points_plotted.png}
	\caption{نقاطی از صفحه فاز که از آن‌ها داده‌گیری انجام شده است به رنگ سفید درآمده‌اند.}
	\label{fig:if_g_d_phase_rotational_space_points_plotted}
\end{figure}

\section{ساده} \label{appendix:phase_sampling_simple}
\begin{figure}[h]
	\includegraphics{../scripts/all_neurons_model_in_one_place/Non_repulsive_rotational_ensembles/N10000_T100_I9.5_13.5_v1.0/sigma_phase_space_contour_points_plotted_alpha20.png}
	\caption{نقاطی از صفحه فاز که از آن‌ها داده‌گیری انجام شده است به رنگ سفید درآمده‌اند.}
	\label{fig:if_g_d_phase_space_Non_repulsive_points_plotted}
\end{figure}