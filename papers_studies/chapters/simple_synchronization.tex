\قسمت{شبکه‌ نورون‌های ساده}
حل مسئله‌ی مدل چرخنده‌ بسیار دشوار است و تا تاریخ نوشتن این بند، راه‌حلی تحلیلی برای توصیف گذرفاز آن نیافته‌ایم. علت این موضوع هم حضور جمله‌ی غیرخطی $- cos(\theta)$ در جمله‌ی برهم‌کنش‌های آن‌هاست. حال که با ابعاد دشوار مسئله روبرو شده‌ایم؛ اجازه دهید که زمین بازی خود را عوض کنیم.\\
می‌پرسیم که آیا کیفیت گذرفاز از ناهم‌گامی به هم‌گامی به این جمله وابسته است؟ بی‌تردید پاسخ این سوال را نخواهیم فهمید؛ مگر آن که شبکه‌ی جدیدی مطابق درخواست خود ابداع و شبیه‌سازی کنیم.

\begin{tcolorbox}
	\begin{equation}
		\begin{cases}
			\dot{\theta_i}=I_i  - g E, \hspace{2ex}  \theta_i \leq \pi \\
			\dot{E} = M - \alpha E\\
			\dot{M} = -  \alpha M + \frac{ \alpha^{2} }{N} \sum_{n|tـn<t} \delta(t - t_n - t_d)
		\end{cases}
	\end{equation}
	\begin{enumerate}[-]
		\item $\theta_i$:
		مشخص کننده‌ی فاز هر نورون. این فاز میان دو لبه در حال حیات است. کوچکترین کران بالای آن همان حالت آستانه در $\pi$ است و بزرگترین کران پایین آن نگه‌دارنده‌ای است که از ریزش نورون‌ها جلوگیری می‌کند.
		\item $E$:
		میدانی است که شدت فعالیت شبکه را نشان می‌دهد.
		\item $M$:
		یک پارامتر فرعی که در حل معادله دیفرانسیل مرتبه دوم به دو معادله‌ی تحول مرتبه اول ما را یاری کرده است.
	\end{enumerate}
\end{tcolorbox}

همچنین دقت کنیم که اگر چه این مدل کاهش یافته‌ای از مدل چرخنده است اما در صورت کاستن مدل انباشت‌وشلیک هم به همین جملات برهم‌کنشی می‌رسیدیم. تنها تفاوت در آن می‌شد که فاصله‌ی بین حالت تیزه ($\pi$) و بازنشانی (صفر) در حالت ابداعی $\pi$ برابر مدل کاسته‌شده‌ی انباشت‌وشلیک می‌شد.

\زیرقسمت{شبیه‌سازی}
برای مدل توصیف شده‌ی بالا شبیه‌سازی خود را با تنظیمات زیر به اجرا گذاشتیم. 
\begin{tcolorbox}[colback=green!5!white,colframe=green!75!black]
	\begin{enumerate}[*]
		\item
		$\alpha = 20\, s^{-1}$
		\item
		جریان‌های تصادفی خارجی نورون‌ها از اعضای بازه‌ی $(9.5,13.5)$ انتخاب می‌شوند.
		\item
		$N = 10000$
		\item
		$t_d = 0.1\, s$ 
	\end{enumerate}
\end{tcolorbox}

\زیرقسمت{نتایج}
\زیرزیرقسمت{در جستجوی تغییرفاز}
قابل توجه است که کیفیت تغییرفاز با حذف جمله‌ی ذکر شده تغییر نکرد و تنها مکان و ارتفاع انحراف از معیار جریان داخلی است که دست خور تغییر شده است - شکل \ref{fig:sigma_non_repulsive}.
\begin{figure}
	\centering
	\includegraphics[width = 10 cm]{../scripts/all_neurons_model_in_one_place/Non_repulsive_rotational_ensembles/N10000_T100_I9.5_13.5/sigma_g_0.1_65.png}
	\caption{پهنای جریان یک سامانه ساده با ده هزار نورون}
	\label{fig:sigma_non_repulsive}
\end{figure}