\فصل{شبکه‌ نورون‌های ساده}
	\label{chap:simple_non_repulsive}
حل مسئله‌ی مدل چرخنده‌ بسیار دشوار است و تا تاریخ نوشتن این بند، راه‌حلی تحلیلی برای توصیف گذرفاز آن نیافته‌ایم. علت این موضوع هم حضور جمله‌ی غیرخطی $- cos(\theta)$ در جمله‌ی برهم‌کنش‌های آن‌هاست. حال که با ابعاد دشوار مسئله روبرو شده‌ایم؛ اجازه دهید که زمین بازی خود را عوض کنیم.\\
می‌پرسیم که آیا کیفیت گذرفاز از ناهم‌گامی به هم‌گامی به این جمله وابسته است؟ بی‌تردید پاسخ این سوال را نخواهیم فهمید؛ مگر آن که شبکه‌ی جدیدی مطابق درخواست خود ابداع و شبیه‌سازی کنیم. این مدل را در جعبه‌ی زیر تعریف کرده‌ایم. عملا تنها کاری که کرده‌ایم حذف جمله‌ی غیر خطی است. به این ترتیب مدل بسیار ساده شده است.

\begin{tcolorbox}
	\begin{equation}
		\begin{cases}
			\dot{\theta_i}=I_i  - g E, \hspace{2ex}  \theta_i \leq \pi \\
			\dot{E} = M - \alpha E\\
			\dot{M} = -  \alpha M + \frac{ \alpha^{2} }{N} \sum_{n|tـn<t} \delta(t - t_n - t_d)
		\end{cases}
	\end{equation}
	\begin{enumerate}[-]
		\item $\theta_i$:
		مشخص کننده‌ی فاز هر نورون. این فاز میان دو لبه در حال حیات است. کوچکترین کران بالای آن همان حالت آستانه در $\pi$ است و بزرگترین کران پایین آن نگه‌دارنده‌ای است که از ریزش نورون‌ها جلوگیری می‌کند.
		\item $E$:
		میدانی است که شدت فعالیت شبکه را نشان می‌دهد.
		\item $M$:
		یک پارامتر فرعی که در حل معادله دیفرانسیل مرتبه دوم به دو معادله‌ی تحول مرتبه اول ما را یاری کرده است.
	\end{enumerate}
\end{tcolorbox}

همچنین دقت کنیم که اگر چه این مدل کاهش یافته‌ای از مدل چرخنده است اما در صورت کاستن مدل انباشت‌وشلیک هم به همین جملات برهم‌کنشی می‌رسیدیم. تنها تفاوت در آن می‌شد که فاصله‌ی بین حالت تیزه ($\pi$) و بازنشانی (صفر) در حالت ابداعی $\pi$ برابر مدل کاسته‌شده‌ی انباشت‌وشلیک می‌شد.

\قسمت{شبیه‌سازی}
برای مدل توصیف شده‌ی بالا شبیه‌سازی خود را با تنظیمات زیر به اجرا گذاشتیم. 
\begin{tcolorbox}[colback=green!5!white,colframe=green!75!black]
	\begin{enumerate}[*]
		\item
		$\alpha = 20\, s^{-1}$
		\item
		جریان‌های تصادفی خارجی نورون‌ها از اعضای بازه‌ی $(9.5,13.5)$ انتخاب می‌شوند.
		\item
		$N = 10000$
		\item
		$t_d = 0.1\, s$ 
	\end{enumerate}
\end{tcolorbox}

\قسمت{نتایج}
\زیرقسمت{در جستجوی تغییرفاز}
قابل توجه است که کیفیت تغییرفاز با حذف جمله‌ی ذکر شده تغییر نکرد و تنها مکان و ارتفاع انحراف از معیار جریان داخلی است که دست خور تغییر شده است - شکل \ref{fig:sigma_non_repulsive}. نکته‌ی جالب‌تر اینجاست که صفحه‌ی فاز هم تقریبا همان رنگ‌آمیزی را دارد که در برای نورون‌های دیگر دیدیم. به جهت اهمیت نکات پیشین اجازه بدهید ملاحظات را یک بار دیگر اینجا آوریم:\\

\begin{enumerate}[1.]
	\item 
	به نظر می‌رسد که با افزایش زمان تاخیر و ضریب تاثیر همگامی قدرت پیدا می‌کند
	\item 
	برای ضریب‌تاثیر یک گذرفاز کاملا ناگهانی و برای تاخیر زمانی که گذرفازی ملایم!
	\item 
	اگر چه تاخیر در جابجایی ضریب‌تاثیر بحرانی تغییری ایجاد نکرده است اما هم‌گامی را قدرت می‌بخشد.
	\item
	اگر ضریب‌تاثیر را بسیار بزرگ کنیم و فاصله‌ی زیادی از ضریب‌تاثیر بحرانی بگیریم؛ شدت هم‌گامی ضعیف می‌شود. این نکته می‌تواند به علت رشد جمعیت نورون‌های خاموش باشد که از تیزه‌زدن محروم می‌شوند و نمی‌توانند خیزهای بلندی را به جریان داخلی القا کنند.
\end{enumerate}



\begin{figure}
	\begin{subfigure}{0.5\textwidth}
		\centering
		\includegraphics[width = \textwidth]{../scripts/all_neurons_model_in_one_place/Non_repulsive_rotational_ensembles/N10000_T100_I9.5_13.5/sigma_g_0.1_65.png}
		\caption{پهنای جریان یک سامانه ساده با ده هزار نورون}
		\label{fig:sigma_non_repulsive}
	\end{subfigure}
	\hfill
	\begin{subfigure}{0.5\textwidth}
		\includegraphics[width = \textwidth]{../scripts/all_neurons_model_in_one_place/Non_repulsive_rotational_ensembles/N10000_T100_I9.5_13.5_v1.0/sigma_phase_space_contour_alpha20.png}
		\caption{
			صفحه‌ی فاز شبکه نورون‌های ساده: پیوست:
			\ref{appendix:phase_samplingـrotational}
		}
		\label{fig:if_g_d_phase_space_Non_repulsive}
	\end{subfigure}
\end{figure}

\زیرقسمت*{تاملی برای گام بعدی}
به نظر می‌آید قدمی که به سمت حذف جمله‌ی غیرخطی داشتیم کاملا مفید بود. زیرا مدل را با حفظ خاصیت گذر فاز، ساده‌تر کرد.  به راستی چه اتفاقی دارد در پس پرده می‌افتد؟ سامانه‌ی هم‌گام با ناهم‌گام چه تفاوت‌هایی دارد؟ پاسخ به این سوال‌ها با مطالعه‌ی اعداد خروجی کاری دشوار است زیرا همواره همراه با حدس و گمان ذهنی است. بیاید قدم بعدی را به سمت «تصویرسازی» برداریم.
