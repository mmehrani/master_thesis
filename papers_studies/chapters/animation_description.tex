\فصل{تصویرسازی سامانه‌ها}
از آنچه که اشکال صفحه‌ی قبل در مورد سامانه روایت می‌کنند؛ می‌توان تنها «حدس» زد که در پس پرده [جعبه سیاه] چه می‌گذرد. به همین دلیل برآن شدم تا روشی برای به تصویر کشیدن سامانه ابداع کنم تا از لحظه‌لحظه‌ی سامانه با خبر شوم. شکل \ref{fig:if_animation_plot}

\begin{figure}[h]
	\centering
	\includegraphics[width =\textwidth]{figs/IF/IF_phase_space-Model.png}
	\caption{تصویر فضای فاز سامانه نورونی انباشت‌وشلیک}
	\label{fig:if_animation_plot}
\end{figure}

پویایی شکل \ref{fig:if_animation_plot} به ما نشان خواهد داد که چگونه سامانه در زمان متحول می‌شود. هر نقطه در این صفحه نمایانگر حالت یک نورون است. محور افقی نشان دهنده‌ی جریان ثابت خارجی است که به هر نورون در ابتدا متصل کرده‌ایم و محور عمودی نشان دهنده‌ی پتانسیل نورون است.\\
طبق توصیفی که از پویایی سامانه‌ی خود داریم؛ توقع داریم که نورون‌هایی که از آستا