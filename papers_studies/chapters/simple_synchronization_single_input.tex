
\زیرقسمت{بازی از نو(سامانه‌ی تک‌جریان)}
بنظر نمی‌آید که معادلات ما از این طریق حل شوند. بیاید یک طریق دیگر در پیش گیریم.

\begin{meeting}
	
	\speak{استاد}
	محسن! بیا مسئله را باز هم ساده‌تر کنیم. به جای آن که یک پهنای جریان بگیریم؛ فقط و فقط یک جریان را در سامانه قرار دهیم. آنگاه ببینیم باز هم هم‌گامی خواهیم دید؟\\
	پیشنهاد بعدی این که تیزه‌ها را باریک و بدون پهنا درنظر بگیر
	($\alpha \rightarrow \infty$)
	امیدوارم در این حالت مسئله حل شود.
	\medskip
	\speak{محسن}
	مسئله شاید کمی عوض شود. زیرا جریان مهاری برآمده از نورون‌هایی با جریان بالا روی پتانسیل نورون‌های پایین‌تر هم تاثیر می‌گذارد.
	\speak{استاد}
	می‌دانم. اما از همین سامانه تک‌جریانی باید دربیاید. وقتی یکی را حل کنیم بقیه را می‌توانیم از کنار هم قرار دادن این زیرسامانه محاسبه کنیم.
	\speak{محسن}
	خیلی هم خوب! چشم انجام می‌شود.
	\speak{استاد}
	فردا می‌توانی بیایی و حضوری باهم جلسه داشته باشیم؟
	\speak{محسن}
	بله حتما خدمت خواهم رسید.
	\attrib{اتاق انجمن علمی، سه‌شنبه عصر ۳۰ فروردین}
	
\end{meeting}
خروجی این مکالمات و چند جلسه پشت سرهم در ادامه‌ی این بخش خواهد آمد.\\

بیاید مجدد رابطه‌ی 
\ref{eq:simple_network}
که جریان را در سامانه گزارش می‌داد برای سامانه‌ی جدید بازنویسی کنیم. با این تفاوت که $\alpha$ را به بینهایت سوق داده‌ایم و تیزه‌ها کاملا باریک هستند.
\begin{align}
	E(t) = \frac{n(\pi,t-d)}{N} \cdot \big[ a - g E(t-d) \big]
\end{align}

\زیرزیرقسمت{حالت‌پایا}
برای این سامانه میدان حالت پایا به صورت زیر قابل توصیف است:
\begin{align}
	E_0 &= \frac{n}{N} \cdot \big[ a - g E_0 \big] = \frac{1}{2\pi}\big[ a - g E_0 \big]\\
	\Rightarrow  & E_0 = \frac{a}{2\pi + g}
\end{align}

\زیرزیرقسمت{اختلال از حالت پایا}
حال فرض کنیم که جریان به اندازه‌ای کوچک از حالت پایای خود منحرف ‌شود.
$E = E_0 + \epsilon$
علاقه‌مندیم که سامانه در زمان‌های بعدی چگونه رفتار خواهد کرد. آیا این اختلال به طریقی هضم خواهد شد و یا بزرگ‌تر می‌شود و هماره سامانه را از حالت پایا دور خواهد کرد؟

\begin{align}
	E(t+d) &= \frac{1}{2\pi} \big[ a - g E(t) \big]\\
	&=  \frac{1}{2\pi} \big[ a - g (E_0 + \epsilon) \big]\\
	&=  \frac{1}{2\pi} \big[ a - g E_0 \big] - \frac{g\epsilon}{2\pi}\\
	&= E_0 - \frac{g\epsilon}{2\pi} \label{eq:Non_repulsive_first_step_field}
\end{align}
با ادامه‌ی همین روند می‌توانیم به این نتیجه برسیم که در گام‌های بعدی سامانه چگونه رفتار خواهد کرد(شکل \ref{fig:non_repulsive_single_input_perturbation}): 
\begin{align}
	E(t + nd) &= \frac{1}{2\pi} \big[ a - g E(t+(n-1)d) \big]\\
	&= \frac{a}{2\pi} \big[ 1 - \frac{g}{2\pi} + (\frac{g}{2\pi})^2 + ... + (\frac{-g}{2\pi})^{n - 1} \big] + (\frac{-g}{2\pi})^{n} (E_0 + \epsilon)
	\label{eq:geometric_sum}\\
	&= \frac{a}{2\pi} \frac{1 - (\frac{-g}{2\pi})^{n} }{ 1 - (\frac{-g}{2\pi})} + (\frac{-g}{2\pi})^{n} (E_0 + \epsilon)\\
	&= \frac{a}{2\pi + g} - \frac{a}{2\pi} \cdot \frac{ (\frac{-g}{2\pi})^{n} }{ 1 - (\frac{-g}{2\pi})} + (\frac{-g}{2\pi})^{n} (E_0 + \epsilon)\\
	&= \frac{a}{2\pi + g} - \frac{a}{2\pi + g} \cdot (\frac{-g}{2\pi})^{n} + (\frac{-g}{2\pi})^{n} (E_0 + \epsilon)\\
	&= E_0 + (\frac{-g}{2\pi})^n \cdot \epsilon\\
	&= E_0 + (\frac{-g}{2\pi})^{\floor{t/d} + 1} \cdot \epsilon
\end{align}

\begin{figure}[h]
	\centering
	\includegraphics[width =0.8\textwidth]{../papers_studies/figs/Perturbation problem-Model.png}
	\caption{
		تحول میدان سامانه‌ای تک‌جریان و با تیزه‌های پهن از اخنلالی کوچک از حالت پایا (شکل اشتباه است)
	}
	\label{fig:non_repulsive_single_input_perturbation}
\end{figure}

پرواضح است که که اگر ضریب تاثیر از مقدار 
$2\pi$
کمتر باشد؛ این مجموع همگراست و اختلال در سامانه هضم خواهد شد. در صورتی اگر بیشتر باشد؛ واگرا خواهد بود. این مقدار بنظر همان گذرفازی است که مدت‌هاست به دنبال آن می‌گردیم. پس موفق شدیم که برای سامانه‌ی تک‌جریان نقطه‌ی گذرفاز را محاسبه کنیم. کمتر باشد؛ این مجموع همگراست و اختلال در سامانه‌ هضم خواهد شد. در صوتی اگر بیشتر باشد؛ واگرا خواهد بود.\\
اگر چه می‌توانستیم مسئله را از طریق دیگری نیز درپیش بگیریم. از معادله‌ی 
\ref{eq:Non_repulsive_first_step_field}
اختلاف جدیدی را که با میدان پایا پیدا شده است را به عنوان 
$\epsilon'$
تعریف کنیم و سپس میدان بعدی را از ضرب مجدد عامل
$ - g/2\pi$
بدست آوریم. به این ترتیب به طریق ساده‌تر خواهیم داشت:
\begin{align}
	E(t+nd) = E_0 + (\frac{-g}{2\pi})^n \cdot \epsilon
\end{align}

\زیرزیرقسمت{شبیه‌سازی سامانه‌ی تک‌جریان}
حدس می‌زنیم که برای سامانه‌ی یاد شده در قسمت قبل گذر فاز در 
$g = 2\pi$
رخ دهد. پس شبیه‌سازی را بار دیگر با تنظیمات زیر راه‌اندازی می‌کنیم:

\begin{tcolorbox}[colback=green!5!white,colframe=green!75!black]
	\begin{enumerate}[*]
		\item
		$\alpha = 100\, s^{-1}$
		\item
		جریان خارجی متصل به همه‌ی نورون‌ها یکسان و برابر ۹.۵ است.
		\item
		$N = 10000$
		\item
		$t_d = 0.1\, s$ 
		\item 
		کل زمان شبیه‌سازی ۱۰۰ ثانیه در نظرگرفته شده
		\item 
		هر گام زمانی برابر $0.01$ ثانیه است.
	\end{enumerate}
\end{tcolorbox}
به این ترتیب نتیجه‌ی شبیه‌سازی به شکل زیر در آمد:
\begin{figure}[h]
	\centering
	\includegraphics[width =\textwidth]{../scripts/all_neurons_model_in_one_place/Non_repulsive_rotational_ensembles/N10000_T100_I9.5_9.5_v2.0/sigma_g_0_130.png}
	\caption{مشخصه‌ی نظم سامانه ده هزار نورونی تک‌جریان}
	\label{fig:sigma_non_repulsive_single_input}
\end{figure}
متاسفانه این شکستی برای امید به محقق شدن توصیف تحلیلی این سامانه است. زیرا نقطه‌ی گذر فاز کاملا دور از همسایگی عدد ۶ و در همسایگی نزدیکی حول ۳.۵ پیدا شده است. سوالی که ما در این بخش با آن تنها خواهیم ماند این است که راه‌حل پیشین ما از چه جزئیاتی چشم پوشی کرده است؟!

\begin{meeting}
	\speak{محسن}
	استاد! شکل  به این صورت درآمد.
	\medskip
	\speak{استاد}
	خوب اشکال ندارد! باید بررسی کنیم ببینیم مشکل از کجاست.
	\attrib{لحظاتی پیش از شروع جلسه‌ی برخط مقاله‌خوانی روز چهارشنبه ۷ اردیبهشت}
\end{meeting}
\begin{mohsenletter}
	\speak{محسن}
	اورکا!
	
	سلام استاد!
	فکر کنم فهمیدم مشکل چیه.
	
	اگر خاطرتون باشه ما باید ضرب سرعت در چگالی روی آستانه را به عنوان محرکه‌ی میدان E در نظر می‌گرفتیم. چون سامانه کمی با حالت پایا فرق داشت؛ چگالی را یکنواخت و ثابت در نظر می‌گرفتیم به طوری که در همسایگی این حالت هم همچنان چگالی ثابت است.
	
	اما این تقریب صحیح نیست! به محض این که علامت سرعت منفی می‌شود (V<0) چگالی روی مرز به صورت پله‌ای تغییر می‌کند و صفر می‌شود. این به این معنی است که اگر برای محاسبه‌ی میدان اکنون در به تاریخچه‌ی سامانه رجوع می‌کنیم؛ باید در نظر داشته باشیم که سهم این رخداد صفر است.
	
	ما سهام‌های رخدادهایی که در آن‌ها  (V<0) است را زیاد شمرده‌ایم و باید حذف شوند.
	
	این نکته به نظر بخشی از مشکل ماست هنوز روی بقیه استدلال دارم کار می‌کنم،
	
	ارادتمند شما\\
	محسن
		\medskip
	\speak{استاد}
	سلام
	
	اگر شرایط اولیه رو این طوری بدیم که مکان همه تصادفی و تاریخچه‌ هم این طور که تا قبل t=0 سرعت‌ها همه مثل هم و یه کم متفاوت با سرعت تعادل، اون وقت چی؟
	\attrib{شنبه ۱۰ اردیبهشت}
\end{mohsenletter}

صحبتی که با استاد مطرح کردم؛ صحیح بود اما نه کاملا صحیح! حدس استاد این است که اگر محور آستانه حول نقطه‌ی گذرفاز خالی شده است به دلیل نامیزان بودن شرایط اولیه است. من حدس خودم و ایشان را مورد بررسی قرار دادم و به این نتیجه رسیدم که گذرفاز در نقطه‌ای رخ می‌دهد که صرفا نورون‌ها کند می‌شوند و برنمی‌گردند؛ یعنی سرعتشان همچنان مثبت است و اندازه‌ی آن کمتر می‌شود اما منفی نمی‌شود. پس حدس من صحیح نبود.\\
پس کنجکاو می‌شویم که نمایش تمام عیاری از صفحه‌ی فاز داشته باشم و بتوانیم شمایل آنچه را که در سامانه رخ داده؛ به تصویر بکشیم.

\begin{figure}[h]
	\begin{subfigure}[b]{0.5\textwidth}
		\centering
		\includegraphics[width = \textwidth]{../scripts/all_neurons_model_in_one_place/Non_repulsive_rotational_ensembles/N10000_T100_I9.5_9.5_v2.0/sigma_phase_space_contour_alpha100.png}
		\caption{صفحه‌ی فاز نورون تک جریان با پهنای تیزه‌ی 
				$\alpha = 100$}
		\label{fig:non_repulsive_single_input_sigma_phase_space_alpha100}
	\end{subfigure}
	\hfill
	\begin{subfigure}[b]{0.5\textwidth}
		\centering
		\includegraphics[width =  \textwidth]{../scripts/all_neurons_model_in_one_place/Non_repulsive_rotational_ensembles/N10000_T100_I9.5_9.5_v2.0/sigma_phase_space_contour_alpha20.png}
		\caption{صفحه‌ی فاز نورون تک جریان با پهنای تیزه‌ی 
		$\alpha = 20$}
		\label{fig:non_repulsive_single_input_sigma_phase_space_alpha20}
	\end{subfigure}
	\label{fig:non_repulsive_single_input_sigma_phase_space}
\end{figure}

چنان که در شکل \ref{fig:non_repulsive_single_input_sigma_phase_space} می‌بینیم در حالتی که تیزه‌ها تقریبا باریک هستند ($\alpha = 100$) تغییر فاز همان است که پیش‌بینی کردیم؛ یعنی در نزدیکی نقطه‌ی 
$g = 2\pi$
رخ می‌دهد و به ازای تمام زمان‌های تاخیر ممکن، همین مقدار می‌ماند اما در تیزه‌های پهن این گذرفاز رفتاری دیگر دارد. زمان‌هایی که تاخیر بسیار بزرگتر از زمان ویژه‌ی تاثیر تیزه‌هاست
($d > \alpha^{-1}$)
گذرفاز در همان نقطه رخ می‌دهد.\\

پس بهتر است این طور جمع بندی‌ کنیم که راه حل
\ref{eq:geometric_sum}
برای حالتی درست است که زمان ویژه تیزه‌ها در مقایسه با زمان تاخیر نسبتا کم باشد.
$d >> \alpha$

حال که مسئله در حالت بسیار ساده حل شد؛ کم‌کم گام‌هایی رو به سمت پیچیده شدن برمی‌داریم. اولین گام آن است که کمیت $\alpha$ را به مجددا به محاسبات خود بازگردانیم تا رابطه‌ی متناظر با
\ref{fig:non_repulsive_single_input_sigma_phase_space}
برای آن بدست آوریم. حدس می‌زنیم که تغییر میدان دیگر مانند شکل 
\ref{fig:non_repulsive_single_input_perturbation}
تیز نباشد و لبه‌هایی نرم‌تر به خود گیرند.
\begin{align}
	E &= E_0 + \epsilon \hspace{2cm} (t<0) \\
	E(t) &= \frac{a}{2\pi} - \frac{g}{2\pi} \int_{0}^{\infty} ds_1 \big( \frac{a}{2\pi} - \frac{g}{2\pi} \int ds_2 E(t - 2d - s_1 - s_2) \alpha^2 s_2 e^{-\alpha s_2} \big) \alpha^2 s_1 e^{-\alpha s_1} \\
	&= \frac{a}{2\pi} \big[ 1 - \frac{g}{2\pi} + (\frac{g}{2\pi})^2 + ... + (\frac{-g}{2\pi})^{n - 1} \big] \\
	&\hspace{3mm}+ (\frac{-g}{2\pi})^{n} \int_{0}^{\infty}\int ... \int E(t - nd - s_1 - s_2 - ... - s_n) \alpha^{2n} s_1 s_2 ... s_n e^{-\alpha(s_1 + s_2 + ... + s_n)} ds_1 ds_2 ... ds_n\\
	&= \mathcal{K}_{a,g}(t)\\
	&\hspace{3mm}+ (\frac{-g}{2\pi})^{n} \int_{0}^{\infty}\int ... \int E(t - nd - s_1 - s_2 - ... - s_n) \alpha^{2n} s_1 s_2 ... s_n e^{-\alpha(s_1 + s_2 + ... + s_n)} ds_1 ds_2 ... ds_n
\end{align}
که در این رابطه 
$\mathcal{K}_{a,g}(t)$
همان پاسخ معادله در حالتی است که تیزه‌ها کاملا باریک هستند. \\
این روند تو در تو تا زمانی ادامه می‌یابد که انتگرال‌ده ما به زمانی که ما آن را به صورت دستی مقید کرده‌ایم نرسد. زیرا از آن گام به بعد رابطه‌ی میدان از دینامیک گام پیشین خود بدست نمی‌آید.\\
آخرین گام این رابطه در زمانی است که بخشی از آن در حالت مقید قرار دارد و بخشی از آن در حالت پویا. پس برای آن که این ملاحظه را اعمال کنیم؛ تغییر متغیر
$r \equiv \sum s_i$
را می‌دهیم تا ورودی تابع میدان را رصد کند. سپس روی مقادیر مختلف $r$ انتگرال‌ها را جمع می‌زنیم.
\begin{align}
	& \int_{0}^{\infty}\int ... \int E(t - nd - s_1 - s_2 - ... - s_n)
	\alpha^{2n} s_1 s_2 ... s_n  e^{-\alpha(s_1 + s_2 + ... + s_n)} ds_1 ds_2 ... ds_n\\
	&= \int_{0}^{\infty}\int ... \int E(t - nd - r) \alpha^{2n} e^{-\alpha(r)} s_1 s_2 ... s_n ds_1 ds_2 ... ds_n \delta(r-\sum s_i) dr
\end{align}
معادل تابع دلتای دیراک برای ادامه‌ی انتگرال‌گیری مناسب‌تر است.
\begin{align}
	&= \int_{0}^{\infty}\int ... \int E(t - nd - r) \alpha^{2n} e^{-\alpha(r)} s_1 s_2 ... s_n ds_1 ds_2 ... ds_n \frac{e^{ip(r-\sum s_i)}}{2\pi} dr dp\\
	&= \int_{0}^{\infty}\int ... \int E(t - nd - r) \alpha^{2n} e^{-\alpha(r)} s_1 s_2 ... s_n ds_1 ds_2 ... ds_n e^{-ip\sum s_i} \frac{e^{ipr}}{2\pi} dr dp\\
	&= \int_{0}^{\infty}\int E(t - nd - r) \alpha^{2n} e^{-\alpha(r)} \frac{e^{ipr}}{2\pi} [\int_0^{\infty} s e^{-ips} ds]^n dr dp
\end{align}
عبارت p در جمله‌ی نهایی باید حتما باید قسمت موهومی منفی داشته باشد تا انتگرال ما قابل تعریف باشد.
\begin{align}
	&= \int_{0}^{\infty}\int E(t - nd - r) \alpha^{2n} e^{-\alpha r} \frac{e^{ipr}}{2\pi} (-p^{-2})^n dp dr
\end{align}
به کمک حساب‌مانده‌ها می‌دانیم که حاصل قسمت $p$ انتگرال برابر مشتق $ 2n-1 $ام از انتگرال‌ده آن است.
\begin{align}
	&= (-1)^n \int_{0}^{\infty} E(t - nd - r) \alpha^{2n} e^{-\alpha r} \frac{2\pi}{2\pi (2n-1)!} i \cdot \frac{\partial^{2n-1} e^{ipr}}{\partial p^{2n-1}}\mid_{p=0} dr\\
	&= \frac{(-1)^n}{(2n-1)!}\int_{0}^{\infty} E(t - nd - r) \alpha^{2n} e^{-\alpha r} i \cdot (ir)^{2n-1} dr\\
	&= \frac{1}{(2n-1)!} \int_{0}^{\infty} E(t - nd - r) \alpha^{2n} e^{-\alpha r} r^{2n-1} dr\\
\end{align}	
همان‌طور که پیشتر گفته بودیم؛ تنها قسمتی از بازه‌ی انتگرال‌گیری ما مربوط به قسمت پویایی است و باید حساب آن را از قسمت قیدی جدا کنیم. این قسمت در بازه‌ی 
$0 < r < t-nd$
وجود دارد.
\begin{align}
	&= \frac{1}{(2n-1)!} \int_{t-nd}^{\infty} E(t - nd - r) \alpha^{2n} e^{-\alpha r} r^{2n-1} dr + \frac{1}{(2n-1)!} \int_{0}^{t - nd}E(t - nd - r) \alpha^{2n} e^{-\alpha r} r^{2n-1} dr	
\end{align}
میدان را در زمان‌های مقید می‌دانیم. از روی آن می‌توانیم میدان در گام اول را نیز محاسبه کنیم.
\begin{align}
	&= \frac{1}{(2n-1)!} \int_{t-nd}^{\infty} (E_0 + \epsilon) \alpha^{2n} e^{-\alpha r} r^{2n-1} dr + \frac{1}{(2n-1)!} \int_{0}^{t - nd}(E_0 - g\epsilon/2\pi) \alpha^{2n} e^{-\alpha r} r^{2n-1} dr\\
	&= \frac{1}{(2n-1)!} (E_0 + \epsilon) \int_{0}^{\infty} \alpha^{2n} e^{-\alpha r} r^{2n-1} dr + \frac{1}{(2n-1)!} ( - \epsilon - g\epsilon/2\pi) \int_{0}^{t - nd} \alpha^{2n} e^{-\alpha r} r^{2n-1} dr	
\end{align}
اگر بازه‌ی انتگرال اول از صفر شروع می‌شد؛ آنگاه می‌توانستیم آن را تبدیل به تابع گاما کنیم. اگر چه صورت کنونی آن به این شکل نیست اما می‌توانیم با اضافه کردن ادامه‌ی بازه به انتگرال و کم کردن آن از انتگرال دیگر، خواسته‌ی خود را برآورده کنیم.
\begin{align}
	&= \frac{1}{(2n-1)!} (E_0 + \epsilon) (2n - 1)! + \frac{1}{(2n-1)!} ( - \epsilon - g\epsilon/2\pi) \int_{0}^{t - nd} \alpha^{2n} e^{-\alpha r} r^{2n-1} dr\\
	&= (E_0 + \epsilon) + \frac{1}{(2n-1)!} ( - \epsilon - g\epsilon/2\pi) \gamma(2n,\alpha( t-nd))
\end{align}
حال که جمله‌ی درخواستی خود را محاسبه کردیم؛ آن را در کنار جملات قبلی قرار می‌دهیم تا در نهایت برای میدان داشته باشیم:
\begin{align}
	E(t) &= \mathcal{K}_{a,g}(t) \\
	&\hspace{3mm}+ (\frac{-g}{2\pi})^n (E_0 + \epsilon) \\
	&\hspace{3mm}+ (\frac{-g}{2\pi})^n \frac{1}{(2n-1)!} ( - \epsilon - g\epsilon/2\pi) \gamma(2n,\alpha( t-nd))\\
	&= \mathcal{K}_{a,g}(t) \\
	&\hspace{3mm}+ (\frac{-g}{2\pi})^n (E_0 + \epsilon)\\
	&\hspace{3mm}+ (\frac{-g}{2\pi})^n ( - \epsilon - g\epsilon/2\pi) \frac{\gamma(2n,\alpha( t-nd))}{(2n-1)!}\\
	&= \mathcal{K}_{a,g}(t) \\
	&\hspace{3mm}+ (\frac{-g}{2\pi})^{\floor{t/d}+1} (E_0 + \epsilon)\\
	&\hspace{3mm}+ (\frac{-g}{2\pi})^{\floor{t/d} + 1} ( - \epsilon - g\epsilon/2\pi) \frac{\gamma(2\floor{t/d},\alpha d( t/d - \floor{t/d}))}{(2n-1)!}\\
	&= E_0 + (\frac{-g}{2\pi})^{\floor{t/d}+1} \cdot \epsilon\\
	&\hspace{3mm}+ (\frac{-g}{2\pi})^{\floor{t/d}+1} ( - \epsilon - g\epsilon/2\pi) \frac{\gamma(2\floor{t/d},\alpha d( t/d - \floor{t/d}))}{(2n-1)!}\\
\end{align}
نتیجه‌ی بدست آمده شامل نکات قابل توجهی است و تا حدودی با شواهد بدست آمده از شبیه‌سازی سازگاری دارد.
\begin{enumerate}
	\item 
	در گام‌های مضرب d، تابع گامای ناقص ما به صورت 
	$ \gamma(2n,0) = 0 $
	درمی‌آید. این باعث می‌شود که فارغ از باریک یا تیزه بودن تیزه‌های ما میدن در این لحظات مطابق 		عبارت زیر باشد:
	\begin{align}
		&= \frac{a}{2\pi} \big[ 1 - \frac{g}{2\pi} + (\frac{g}{2\pi})^2 + ... + (\frac{-g}{2\pi})^{\floor{t/d} - 1} \big] + (\frac{g}{2\pi})^{\floor{t/d}} (E_0 + \epsilon)
	\end{align}
	
	\item 
	این عبارت با معادلاتی که برای تیزه‌های باریک بدست آوردیم سازگاری دارد. زیرا اگر کمیت
	$\alpha$
	را به بینهایت میل دهیم؛ بیشینه‌ی تابع گامای ناقص ما به صورت 
	$ \gamma(2n,\infty) = (2n-1)! $
	درمی‌آید. در این حالت میدان نهایی عبارتی شبیه توصیف ‌کننده‌ی سامانه‌ی تیزه‌های باریک می‌دهد.
	\begin{align}
		&= \frac{a}{2\pi} \big[ 1 - \frac{g}{2\pi} + (\frac{g}{2\pi})^2 + ... + (\frac{-g}{2\pi})^{\floor{t/d} - 1} \big] + (\frac{g}{2\pi})^{\floor{t/d}} (E_0 - g\epsilon/2\pi)
	\end{align}
	
\end{enumerate}

\زیرزیرقسمت{مشکل پابرجا}

متاسفانه مشکل همچنان پابرجاست. پس از تلاش‌های بسیار و زیرورو کردن راه‌حل مشکلات زیادی از میان آن بیرون کشیده شد. هر چند در ابتدا بنظر می‌آمد خودسازگار و جامع باشد.\\
مسئله‌ی مهم آن است که ما بی‌مهابا هر دم میدان را درون خود جاگذاری می‌کنیم. گذشته از این که این کار به ازای تمامی زمان‌های شبیه‌سازی صحیح نیست. همواره وقتی میدان در زمان‌های منفی سیر می‌کند؛ باید حالت مقید را جاگذاری کنیم. این اتفاق در هر مرحله از بازنویسی انتگرال برحسب گام‌های قبلی رخ می‌دهد. پس قاعدتا هر مرحله باید انتگرال خود را به درستی تفکیک کنیم و فقط برای زمان‌های مثبت خود میدان را بر حسب خود بازنویسی کنیم.\\
این بار بگذارید که به جای عبارت
$\rho_{\pi} a = a/2\pi$
و
$\rho_{\pi} g =  g/2\pi$
بگذاریم:
$\hat{g}, \hat{a}$

\begin{align}
	E &= E_0 + \epsilon \hspace{2cm} (t<0) \\
	E(t) &= \hat{a} - \hat{g} \int_{0}^{\infty} ds_1 E(t - d - s_1) \alpha^2 s_1 e^{-\alpha s_1} \\
	&= \hat{a} - \hat{g} \int_{t - d}^{\infty} ds_1 E(t - d - s_1) \alpha^2 s_1 e^{-\alpha s_1} \\
	&\hspace{1 cm} - \hat{g} \int_{0}^{t - d} ds_1 E(t - d - s_1) \alpha^2 s_1 e^{-\alpha s_1} \\
	&= \hat{a} - \hat{g} \int_{t - d}^{\infty} ds_1 ( E_0 + \epsilon ) \alpha^2 s_1 e^{-\alpha s_1}\\
	& \hspace{1 cm} - \hat{g} \int_{0}^{t - d} ds_1 \big( \hat{a} - \hat{g} \int_{0}^{\infty} ds_2 E(t - 2d - s_1 - s_2) \alpha^2 s_2 e^{-\alpha s_2} \big) \alpha^2 s_1 e^{-\alpha s_1} \\
	&= \hat{a} - \hat{g} \int_{t - d}^{\infty} ds_1 ( E_0 + \epsilon ) \alpha^2 s_1 e^{-\alpha s_1}\\
	& \hspace{1 cm} - \hat{g} \hat{a} \int_{0}^{t - d} ds_1 \alpha^2 s_1 e^{-\alpha s_1}\\
	& \hspace{1 cm} - \hat{g} \int_{0}^{t - d} \int_{0}^{\infty} ds_1 ds_2 E(t - 2d - s_1 - s_2) \alpha^2 s_2 e^{-\alpha s_2} \alpha^2 s_1 e^{-\alpha s_1} \\
	&= \hat{a} - \hat{g} \int_{t - d}^{\infty} ds_1 ( E_0 + \epsilon ) \alpha^2 s_1 e^{-\alpha s_1}\\
	& \hspace{1 cm} - \hat{g} \hat{a} \int_{0}^{t - d} ds_1 \alpha^2 s_1 e^{-\alpha s_1}\\
	& \hspace{1 cm} + (-\hat{g})^2 \int_{0}^{t-d} \int_{t - 2d - s_1}^{\infty} ds_1 ds_2 E(t - 2d - s_1 - s_2) \alpha^2 s_2 e^{-\alpha s_2} \alpha^2 s_1 e^{-\alpha s_1} \\
	& \hspace{1 cm} + (-\hat{g})^2 \int_{0}^{t - d} \int_{0}^{t - 2d - s_1} ds_1 ds_2 E(t - 2d - s_1 - s_2) \alpha^2 s_2 e^{-\alpha s_2} \alpha^2 s_1 e^{-\alpha s_1} \\
	&= \hat{a} - \hat{g} \int_{t - d}^{\infty} ds_1 ( E_0 + \epsilon ) \alpha^2 s_1 e^{-\alpha s_1}\\
	& \hspace{1 cm} - \hat{g} \hat{a} \int_{0}^{t - d} ds_1 \alpha^2 s_1 e^{-\alpha s_1}\\
	& \hspace{1 cm} +(-\hat{g})^2 \int_{0}^{t-d} \int_{t - 2d - s_1}^{\infty} ds_1 ds_2 (E_0 + \epsilon) \alpha^2 s_2 e^{-\alpha s_2} \alpha^2 s_1 e^{-\alpha s_1} \\
	& \hspace{1 cm} +(-\hat{g})^2 \int_{0}^{t - d} \int_{0}^{t - 2d - s_1} ds_1 ds_2 E(t - 2d - s_1 - s_2) \alpha^2 s_2 e^{-\alpha s_2} \alpha^2 s_1 e^{-\alpha s_1}
\end{align}

\begin{landscape}
با ادامه‌ی این روند زنجیروار می‌توانیم حدس بزنیم که خانواده‌ای از جمع جملات متفاوت خواهیم داشت که به صورت زیر قابل نوشتن هستند:
\begin{align}
	E(t) &= \hat{a} + \hat{a} \sum^{n - 1}_{i=1} (-\hat{g})^i \int_{0}^{t - d} \int_{0}^{t - 2d - s_1} \int_{0}^{t - 3d - s_1 - s_2} ... \int_{0}^{t - id - s_1 - ... s_{i-1}} \Pi^i_j \alpha^2 s_j e^{-\alpha s_j} ds_j \\
	&\hspace{1 cm} + (E_0 + \epsilon) \sum^{n}_{i=1} (-\hat{g})^i \int_{0}^{t - d} \int_{0}^{t - 2d - s_1} \int_{0}^{t - 3d - s_1 - s_2} ... \int_{t - id - s_1 - ... s_{i-1}}^{\infty} \Pi^i_j \alpha^2 s_j e^{-\alpha s_j} ds_j \\
	&\hspace{1 cm} + (E_0 - \epsilon \hat{g}) (-\hat{g})^n \int_{0}^{t - d} \int_{0}^{t - 2d - s_1} \int_{0}^{t - 3d - s_1 - s_2} ... \int_{0}^{t - nd - s_1 - ... s_{n-1}} \Pi^n_j \alpha^2 s_j e^{-\alpha s_j} ds_j \\
	&= \hat{a} \sum^{n - 1}_{i=0} (-\hat{g})^i \int_{0}^{t - d} \int_{0}^{t - 2d - s_1} \int_{0}^{t - 3d - s_1 - s_2} ... \int_{0}^{t - id - s_1 - ... s_{i-1}} \Pi^i_j \alpha^2 s_j e^{-\alpha s_j} ds_j \\
	& (E_0 + \epsilon) \sum^{n}_{i=1} (-\hat{g})^i \int_{0}^{t - d} \int_{0}^{t - 2d - s_1} \int_{0}^{t - 3d - s_1 - s_2} ... \int_0^{t - id - s_1 - ... - s_{i-1}} (1 - \int_{0}^{t - id - s_1 - ... - s_{i-1}} \Pi^i_j \alpha^2 s_j e^{-\alpha s_j} ds_j )\\
	&\hspace{1 cm} + (E_0 - \epsilon \hat{g}) (-\hat{g})^n \int_{0}^{t - d} \int_{0}^{t - 2d - s_1} \int_{0}^{t - 3d - s_1 - s_2} ... \int_{0}^{t - nd - s_1 - ... s_{n-1}} \Pi^n_j \alpha^2 s_j e^{-\alpha s_j} ds_j\\
	&= \hat{a} \sum^{n - 1}_{i=0} (-\hat{g})^i I_i\\
	&\hspace{1 cm} + (E_0 + \epsilon) \sum^{n}_{i=1} (-\hat{g})^i [I_{i-1} - I_{i}]  \\
	&\hspace{1 cm} + (E_0 - \epsilon \hat{g}) (-\hat{g})^n I_n \\
	&= E_0 - \epsilon g + \sum^{n}_{i=1} - (-g)^i \cdot \epsilon(1+g) I_{i}
\end{align}
\end{landscape}
که در آن عبارت 
$I_n$
برابر است با:

\begin{align}
	I_n = \int_{0}^{t - d} \int_{0}^{t - 2d - s_1} \int_{0}^{t - 3d - s_1 - s_2} ... \int_{0}^{t - nd - s_1 - ... s_{n-1}} \Pi^n_j \alpha^2 s_j e^{-\alpha s_j} ds_j
	\label{eq:spikes_sandwiches}
\end{align}

\زیرزیرقسمت{نقطه‌ی گذرفاز پیشنهادی}
این محاسبات باید بتوانند نقطه‌ی گذرفاز را پیش‌بینی کنند. برای این منظور اجازه دهید تا بررسی کنیم که روند فراز و فرود میدان به چه صورت تغییر می‌کند و هر مرحله اندازه‌ی جمله‌ای که به سری اضافه می‌شود چیست. پس تعریف می‌کنیم:
\begin{align}
	\tilde{G}_{n} := \tilde{E}_{n} - \tilde{E}_{n-1}
\end{align}
پس بررسی می‌کنیم:
\begin{align}
	T_{n+1} = \frac{|\tilde{G}_{n+1}|}{|\tilde{G}_{n}|} &= |-\hat{g} \cdot \frac{I_{n+1}}{I_n}|
	\label{eq:golden_ratio}
\end{align}
حال همه چیز به نسبت بدست آمده در رابطه‌ی
\ref{eq:golden_ratio}
دارد. اگر این نسبت بزرگتر از یک باشد؛ آنگاه همگامی اتفاق می‌افتد و اگر کوچکتر از یک بشود ناهمگامی داریم.\\
پیش از ادامه شایان ذکر است که در مرحله‌ای که تیزه‌ها بسیار باریک هستند؛ همه‌ی عبارات 
$I_n$
برابر یک هستند و نسبت تصاعد بدست آمده در رابطه‌ی 
\ref{eq:golden_ratio}
همان g خواهد شد که در قسمت‌های پیشین محاسبه کردیم. پس این نتیجه‌ی ما با نتایج قبلی همخوانی بسیار خوبی دارد.\\

حال ادامه می‌دهیم و می‌پرسیم که رفتار عبارت‌های 
$I_n$
به چگونه است و آیا قادر هستند که نقطه‌ی گذرفاز را جابجا کنند یا خیر. همان‌طور که در رابطه‌ی 
\ref{eq:spikes_sandwiches}
دیدیم؛ محاسبه‌ی این انتگرال‌ها بسیار دشوار است اما می‌توانیم به محاسبه‌ی چند جمله‌ی اول 
$g_{T_n}$
بسنده کنیم.
\begin{align}
	1 &= \hat{g} \cdot \frac{I_1}{I_0} = \hat{g} \gamma(2,\alpha d)\\
	\Rightarrow \hat{g}_{T_1} &= \frac{1}{\gamma(2,\alpha d)}\\
	1 &= \hat{g} \cdot \frac{I_2}{I_1} = \hat{g} \cdot \frac{\int_0^{d} \gamma(2,2\alpha (d-s_1)) \alpha^2 s_1 e^{-\alpha s_1} ds_1 }{\gamma(2,\alpha d)}\\
	&= \hat{g} \cdot \frac{1 - e^{- \alpha d}( (\alpha d)^3 + 3(\alpha d)^2 + 6(\alpha d) + 6)/6}{\gamma(2,\alpha d)}\\
	\Rightarrow \hat{g}_{T_2} &= \frac{\gamma(2,\alpha d)}{1 - e^{- \alpha d}( (\alpha d)^3 + 3(\alpha d)^2 + 6(\alpha d) + 6)/6}
\end{align}

\زیرزیرقسمت{هم‌خوانی با هم‌گامی}
حال که موفق شدیم تا حدسی در مورد نقطه‌ی گذرفاز بزنیم نوبت آن است که نتایج خود را با شبیه‌سازی مقایسه کنیم. در شکل
این تطابق موفقیت بزرگی است که توانستیم پس از تلاش‌های فراوان به آن دست پیدا کنیم.
\\

\begin{figure}
	\includegraphics{../scripts/all_neurons_model_in_one_place/Non_repulsive_rotational_ensembles/N10000_T100_I9.5_9.5_v2.0/sigma_phase_space_contour_analytical_trans_points_alpha20.png}
	\caption{مقایسه‌ی نقطه‌ی گذرفاز محاسبه شده و داده‌های شبیه‌سازی}
\end{figure}







