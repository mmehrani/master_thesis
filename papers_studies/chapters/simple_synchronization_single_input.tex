
\زیرقسمت{بازی از نو(سامانه‌ی تک‌جریان)}
بنظر نمی‌آید که معادلات ما از این طریق حل شوند. بیاید یک طریق دیگر در پیش گیریم.

\begin{meeting}
	
	\speak{استاد}
	محسن! بیا مسئله را باز هم ساده‌تر کنیم. به جای آن که یک پهنای جریان بگیریم؛ فقط و فقط یک جریان را در سامانه قرار دهیم. آنگاه ببینیم باز هم هم‌گامی خواهیم دید؟\\
	پیشنهاد بعدی این که تیزه‌ها را باریک و بدون پهنا درنظر بگیر
	($\alpha \rightarrow \infty$)
	امیدوارم در این حالت مسئله حل شود.
	\medskip
	\speak{محسن}
	مسئله شاید کمی عوض شود. زیرا جریان مهاری برآمده از نورون‌هایی با جریان بالا روی پتانسیل نورون‌های پایین‌تر هم تاثیر می‌گذارد.
	\speak{استاد}
	می‌دانم. اما از همین سامانه تک‌جریانی باید دربیاید. وقتی یکی را حل کنیم بقیه را می‌توانیم از کنار هم قرار دادن این زیرسامانه محاسبه کنیم.
	\speak{محسن}
	خیلی هم خوب! چشم انجام می‌شود.
	\speak{استاد}
	فردا می‌توانی بیایی و حضوری باهم جلسه داشته باشیم؟
	\speak{محسن}
	بله حتما خدمت خواهم رسید.
	\attrib{اتاق انجمن علمی، سه‌شنبه عصر ۳۰ فروردین}
	
\end{meeting}
خروجی این مکالمات و چند جلسه پشت سرهم در ادامه‌ی این بخش خواهد آمد.\\

بیاید مجدد رابطه‌ی 
\ref{eq:simple_network}
که جریان را در سامانه گزارش می‌داد برای سامانه‌ی جدید بازنویسی کنیم. با این تفاوت که $\alpha$ را به بینهایت سوق داده‌ایم و تیزه‌ها کاملا باریک هستند.
\begin{align}
	E(t) = \frac{n(\pi,t-d)}{N} \cdot \big[ a - g E(t-d) \big]
\end{align}

\زیرزیرقسمت{حالت‌پایا}
برای این سامانه میدان حالت پایا به صورت زیر قابل توصیف است:
\begin{align}
	E_0 &= \frac{n}{N} \cdot \big[ a - g E_0 \big] = \frac{1}{2\pi}\big[ a - g E_0 \big]\\
	\Rightarrow  & E_0 = \frac{a}{2\pi + g}
\end{align}

\زیرزیرقسمت{اختلال از حالت پایا}
حال فرض کنیم که جریان به اندازه‌ای کوچک از حالت پایای خود منحرف ‌شود.
$E = E_0 + \epsilon$
علاقه‌مندیم که سامانه در زمان‌های بعدی چگونه رفتار خواهد کرد. آیا این اختلال به طریقی هضم خواهد شد و یا بزرگ‌تر می‌شود و هماره سامانه را از حالت پایا دور خواهد کرد؟

\begin{align}
	E(t+d) &= \frac{1}{2\pi} \big[ a - g E(t) \big]\\
	&=  \frac{1}{2\pi} \big[ a - g (E_0 + \epsilon) \big]\\
	&=  \frac{1}{2\pi} \big[ a - g E_0 \big] - \frac{g\epsilon}{2\pi}\\
	&= E_0 - \frac{g\epsilon}{2\pi} 
\end{align}
با ادامه‌ی همین روند می‌توانیم به این نتیجه برسیم که در گام‌های بعدی سامانه چگونه رفتار خواهد کرد:
\begin{align}
	E(t + nd) &= \frac{1}{2\pi} \big[ a - g E(t+(n-1)d) \big]\\
	&= E_0 + \epsilon \sum_n (\frac{-g}{2\pi})^n
	\label{eq:geometric_sum}
\end{align}
پرواضح است که که اگر ضریب تاثیر از مقدار 
$2\pi$
کمتر باشد؛ این مجموع همگراست و اختلال در سامانه هضم خواهد شد. در صورتی اگر بیشتر باشد؛ واگرا خواهد بود. این مقدار بنظر همان گذرفازی است که مدت‌هاست به دنبال آن می‌گردیم. پس موفق شدیم که برای سامانه‌ی تک‌جریان نقطه‌ی گذرفاز را محاسبه کنیم. کمتر باشد؛ این مجموع همگراست و اختلال در سامانه‌ هضم خواهد شد. در صوتی اگر بیشتر باشد؛ واگرا خواهد بود.\\

\زیرزیرقسمت{شبیه‌سازی سامانه‌ی تک‌جریان}
حدس می‌زنیم که برای سامانه‌ی یاد شده در قسمت قبل گذر فاز در 
$g = 2\pi$
رخ دهد. پس شبیه‌سازی را بار دیگر با تنظیمات زیر راه‌اندازی می‌کنیم:

\begin{tcolorbox}[colback=green!5!white,colframe=green!75!black]
	\begin{enumerate}[*]
		\item
		$\alpha = 100\, s^{-1}$
		\item
		جریان خارجی متصل به همه‌ی نورون‌ها یکسان و برابر ۹.۵ است.
		\item
		$N = 10000$
		\item
		$t_d = 0.1\, s$ 
		\item 
		کل زمان شبیه‌سازی ۱۰۰ ثانیه در نظرگرفته شده
		\item 
		هر گام زمانی برابر ۰.۰۱ ثانیه است.
	\end{enumerate}
\end{tcolorbox}
به این ترتیب نتیجه‌ی شبیه‌سازی به شکل زیر در آمد:
\begin{figure}[h]
	\centering
	\includegraphics[width =\textwidth]{../scripts/all_neurons_model_in_one_place/Non_repulsive_rotational_ensembles/N10000_T100_I9.5_9.5_v2.0/sigma_g_0_130.png}
	\caption{مشخصه‌ی نظم سامانه ده هزار نورونی تک‌جریان}
	\label{fig:sigma_non_repulsive_single_input}
\end{figure}
متاسفانه این شکستی برای امید به محقق شدن توصیف تحلیلی این سامانه است. زیرا نقطه‌ی گذر فاز کاملا دور از همسایگی عدد ۶ و در همسایگی نزدیکی حول ۳.۵ پیدا شده است. سوالی که ما در این بخش با آن تنها خواهیم ماند این است که راه‌حل پیشین ما از چه جزئیاتی چشم پوشی کرده است؟!

\begin{meeting}
	\speak{محسن}
	استاد! شکل  به این صورت درآمد.
	\medskip
	\speak{استاد}
	خوب اشکال ندارد! باید بررسی کنیم ببینیم مشکل از کجاست.
	\attrib{لحظاتی پیش از شروع جلسه‌ی برخط مقاله‌خوانی روز چهارشنبه ۷ اردیبهشت}
\end{meeting}
\begin{mohsenletter}
	\speak{محسن}
	اورکا!
	
	سلام استاد!
	فکر کنم فهمیدم مشکل چیه.
	
	اگر خاطرتون باشه ما باید ضرب سرعت در چگالی روی آستانه را به عنوان محرکه‌ی میدان E در نظر می‌گرفتیم. چون سامانه کمی با حالت پایا فرق داشت؛ چگالی را یکنواخت و ثابت در نظر می‌گرفتیم به طوری که در همسایگی این حالت هم همچنان چگالی ثابت است.
	
	اما این تقریب صحیح نیست! به محض این که علامت سرعت منفی می‌شود (V<0) چگالی روی مرز به صورت پله‌ای تغییر می‌کند و صفر می‌شود. این به این معنی است که اگر برای محاسبه‌ی میدان اکنون در به تاریخچه‌ی سامانه رجوع می‌کنیم؛ باید در نظر داشته باشیم که سهم این رخداد صفر است.
	
	ما سهام‌های رخدادهایی که در آن‌ها  (V<0) است را زیاد شمرده‌ایم و باید حذف شوند.
	
	این نکته به نظر بخشی از مشکل ماست هنوز روی بقیه استدلال دارم کار می‌کنم،
	
	ارادتمند شما\\
	محسن
		\medskip
	\speak{استاد}
	سلام
	
	اگر شرایط اولیه رو این طوری بدیم که مکان همه تصادفی و تاریخچه‌ هم این طور که تا قبل t=0 سرعت‌ها همه مثل هم و یه کم متفاوت با سرعت تعادل، اون وقت چی؟
	\attrib{شنبه ۱۰ اردیبهشت}
\end{mohsenletter}

صحبتی که با استاد مطرح کردم؛ صحیح بود اما نه کاملا صحیح! حدس استاد این است که اگر محور آستانه حول نقطه‌ی گذرفاز خالی شده است به دلیل نامیزان بودن شرایط اولیه است. من حدس خودم و ایشان را مورد بررسی قرار دادم و به این نتیجه رسیدم که گذرفاز در نقطه‌ای رخ می‌دهد که صرفا نورون‌ها کند می‌شوند و برنمی‌گردند؛ یعنی سرعتشان همچنان مثبت است و اندازه‌ی آن کمتر می‌شود اما منفی نمی‌شود. پس حدس من صحیح نبود.\\
پس کنجکاو می‌شویم که نمایش تمام عیاری از صفحه‌ی فاز داشته باشم و بتوانیم شمایل آنچه را که در سامانه رخ داده؛ به تصویر بکشیم.

\begin{figure}[h]
	\begin{subfigure}[b]{0.5\textwidth}
		\centering
		\includegraphics[width = \textwidth]{../scripts/all_neurons_model_in_one_place/Non_repulsive_rotational_ensembles/N10000_T100_I9.5_9.5_v2.0/sigma_phase_space_contour_alpha100.png}
		\caption{صفحه‌ی فاز نورون تک جریان با پهنای تیزه‌ی 
				$\alpha = 100$}
		\label{fig:non_repulsive_single_input_sigma_phase_space_alpha100}
	\end{subfigure}
	\hfill
	\begin{subfigure}[b]{0.5\textwidth}
		\centering
		\includegraphics[width =  \textwidth]{../scripts/all_neurons_model_in_one_place/Non_repulsive_rotational_ensembles/N10000_T100_I9.5_9.5_v2.0/sigma_phase_space_contour_alpha20.png}
		\caption{صفحه‌ی فاز نورون تک جریان با پهنای تیزه‌ی 
		$\alpha = 20$}
		\label{fig:non_repulsive_single_input_sigma_phase_space_alpha20}
	\end{subfigure}
	\label{fig:non_repulsive_single_input_sigma_phase_space}
\end{figure}

چنان که در شکل \ref{fig:non_repulsive_single_input_sigma_phase_space} می‌بینیم در حالتی که تیزه‌ها تقریبا باریک هستند ($\alpha = 100$) تغییر فاز همان است که پیش‌بینی کردیم؛ یعنی در نزدیکی نقطه‌ی 
$g = 2\pi$
رخ می‌دهد و به ازای تمام زمان‌های تاخیر ممکن، همین مقدار می‌ماند اما در تیزه‌های پهن این گذرفاز رفتاری دیگر دارد. زمان‌هایی که تاخیر بسیار بزرگتر از زمان ویژه‌ی تاثیر تیزه‌هاست
($d > \alpha^{-1}$)
گذرفاز در همان نقطه رخ می‌دهد.\\

پس بهتر است این طور جمع بندی‌ کنیم که راه حل
\ref{eq:geometric_sum}
برای حالتی درست است که زمان ویژه تیزه‌ها در مقایسه با زمان تاخیر نسبتا کم باشد.
$d >> \alpha$

حال که مسئله در حالت بسیار ساده حل شد؛ کم‌کم گام‌هایی رو به سمت پیچیده شدن برمی‌داریم. اولین گام آن است که کمیت $\alpha$ را به مجددا به محاسبات خود بازگردانیم تا رابطه‌ی متناظر با
\ref{fig:non_repulsive_single_input_sigma_phase_space}
برای آن بدست آوریم.
\begin{align}
	E &= E_0 + \epsilon \hspace{2cm} (t<0) \\
	E(t) &= a - \frac{g}{2\pi} \int_{0}^{\infty} ds_1 \big( a - \frac{g}{2\pi} \int ds_2 E(t - 2d - s_1 - s_2) \alpha^2 e^{-\alpha s_2} \big) \alpha^2 e^{-\alpha s_1} \\
	&= a \big[ 1 - \frac{g}{2\pi} + (\frac{g}{2\pi})^2 + ... + (\frac{-g}{2\pi})^{\floor{t/d} - 1} \big] + (E_0 + \epsilon) (\frac{-g}{2\pi})^{\floor{t/d}}
\end{align}