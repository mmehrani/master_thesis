
\فصل{تلاش برای توصیف}
\label{chap:analytics}

از آنجا که معادلات پاره‌ای شبکه‌های نورونی ما شامل جزئیات بسیار است؛ بدیهی است که نوشتن پاسخ تحلیلی برای توصیف دقیق آن آسان نباشد. در این بخش تلاش می‌کنیم که از جهات متفاوت به مسئله‌ی خود حمله کنیم؛ باشد که بتوانیم صورتی کلی برای مسئله‌ی خود ارائه دهیم. شروع ما با اطلاعات ساده‌ای است که در ابتدا می‌دانیم.\\

می‌دانیم هر نورونی که از حالت $\theta = \pi$ عبور می‌کند [تیزه می‌زند] باعث می‌شود تا سهمی از جریان با کیفیت $p(t):= \alpha^2 t \cdot exp(-\alpha t)$ به جریان درونی کل سامانه $E(t)$ اضافه شود.
\begin{equation}
	E(t) = \frac{1}{N}\int_{0}^{\infty} \int J_a (\pi,t-d-u) da \cdot \alpha^2 u\, e^{-\alpha u} du 
	\label{eq:general_field}
\end{equation}

اما جریان برای هر نورون با ورودی $a$ به طریق زیر است:
\begin{equation}
	J_a (\theta, t) = n_a(\theta,t) \cdot \dot \theta_a
\end{equation}
این رفتار به خوبی نشان می‌دهد جریان فقط در ناحیه‌ی $\theta \leq \pi$ وجود دارد. زیرا ورود نورون به ناحیه‌ی مثبت‌تر را ممنوع کرده‌ایم.  بی‌تردید برای فهمیدن چگونگی تغییر جریان در ناحیه‌های میانی باید از معادله‌ی پخش استفاده کنیم.
\begin{align}
	\frac{\partial n_a}{\partial t} &= - \frac{\partial J_a}{\partial \theta} \label{eq:continuity_relation}\\
	&= - \frac{\partial n_a}{\partial \theta} \cdot \dot \theta_a
\end{align}
\قسمت{حل معادله‌ی شبکه‌ی ساده}
اجازه بدهید تا اولین تلاش خود را از ساده‌ترین نوع شبکه‌ها شروع کنیم. شبکه‌ای که به جز جریان داخلی و جریان تصادفی اولیه ورودی دیگری ندارد. پس خواهیم داشت:
\begin{align}
	\begin{cases}
		E(t) = \frac{1}{N} \int_{0}^{\infty} \int n_a(\pi,t-d-u) \cdot \big[ a - g E(t-d-u) \big] da \cdot \alpha^2 u\, e^{-\alpha u} du \\
		\frac{\partial n_a}{\partial t} = - \frac{\partial n_a}{\partial \theta} \cdot (a - g E(t) )
	\end{cases}
	\label{eq:simple_network}
\end{align}
همان‌طور که از معادلات بالا مشخص است ما با یک دستگاه مواجه هستیم که دستور تغییر E را به خود E در زمان‌های گذشته مربوط کرده است. گذشته از آن که تغییر عبارت
$n_a$
نیز بر طبق معادله‌ی پخش به E وابسته است! همان‌طور که خواننده مدنظر دارد؛ ما با معادلاتی طرف هستیم که بسیار گره‌خورده‌اند.\\

بی‌تردید می‌توان از زوایای متفاوت به مسئله نگاه کرد و راه‌حل‌های زیادی را پیشنهاد داد. ما در اینجا پیشنهاداتی را که در زمان مواجهه اولیه با این دستگاه داشتیم در زیر آوردیم:
\begin{enumerate}[1.]
	\item
انتگرال اول را به صورت بازگشتی در خودش جاگذاری کنیم.
	\item
	از آنجا که میدان به گونه‌ای متناوب عمل می‌کند؛ یک پیشنهاد خوب می‌تواند آن باشد که بسط فوریه‌ی آن را بنویسیم.
	\begin{equation}
		E(t) = \sum c_i \cdot cos(\omega_i t)
	\end{equation}
	که اگر ثابت کنیم تحت شرایطی یکی از ضرایب
	 $c_i$
	 از بقیه ضرایب بزرگتر می‌شود؛ آنگاه مساله‌ی ما حل می‌شود.
	\item
	دشواری مساله از در هم تنیدگی زمانی معادلات برآمده است. مسئله را در حالت آماری بررسی کنیم و حالت پایستار 
	\footnote{حالتی است که کمیت‌های بزرگ مقیاس با زمان تغییر نمی‌کنند.}
	آن را پیدا کنیم. سپس در مرتبه‌ی بالاتر یک نوفه‌ی کوچک به میدان اضافه کنیم و مشاهده کنیم که پاسخ معادلات چگونه تغییر می‌کنند.
\end{enumerate}



\زیرقسمت{روش بازگشتی}
نکته‌ای که برای ما حل معادلات را دشوار می‌کند تبعیت $E$ از خودش است. بگذارید به شیوه‌ای که خود معادله درخواست دارد عمل کنیم. یعنی $E$ را مجددا در سمت راست معادله جاگذاری کنیم. برای راحت‌تر شدن محاسبات ابتدا دو متغیر کمکی زیر را تعریف می‌کنیم:
% abbrivations for calculation of this section
\newcommand{\J}[1]{\mathcal{J}(\pi,#1)}
\newcommand{\N}[1]{\mathcal{N}(\pi, #1)}
\newcommand{\Pexp}[1]{\mathcal{P}(u_{#1})}

\newcommand{\A}[1]{\mathcal{A}(#1)}

%\newcommand{\impact}[1]{u_{#1}\, e^{-\alpha u_{#1}}}
%
\begin{align}
	\J{t-d-u} &\equiv \int n_a(\pi,t-d-u) a \cdot da\\
	\N{t-d-u} &\equiv \int n_a(\pi,t-d-u) \cdot da\\
	\Pexp{} &\equiv \alpha^2 u\, e^{-\alpha u}
\end{align}
عبارت $\J{u}$ به معنای جمع جریان تصادفی نورون‌هایی است که در زمان u در آستانه قرار دارند. همچنین عبارت $\N{u}$ به معنای تعداد همین نورون‌هاست.\\
حال با نمادهای بالا شروع به بازنویسی جملات پیشین می‌کنیم:
\begin{align}
	E(t) &= \frac{1}{N} \int_{0}^{\infty} \J{t-d-u_1} \cdot \Pexp{1} du_1\\
	 &-\frac{g}{N}\int_{0}^{\infty} \N{t-d-u_1} \cdot \Pexp{1} E(t-d-u_1)  du_1
	 \label{eq:field_zeroth_step}
\end{align}
 جمله‌ی اول را می‌توانیم با عبارت دیگری خلاصه‌سازی می‌کنیم:

\begin{equation}
	\A{t} \equiv \frac{1}{N}\int_{0}^{\infty} \J{t-d-u_1} \cdot \Pexp{1} du_1
\end{equation}
 جمله‌ی دوم که حاوی عبارتی از جنس میدان است می‌تواند با معادله‌ای مشابه بازنویسی شود. به این ترتیب برای آن خواهیم داشت:
\begin{align}
	E(t - d - u_1) &= \frac{1}{N} \int_{0}^{\infty} \J{t-2d-u_1-u_2} \cdot \Pexp{2} du_2\\
	&- \frac{g}{N}\int_{0}^{\infty} \N{t-2d-u_1-u_2} \cdot \Pexp{2} E(t-2d-u_1-u_2)  du_2
\end{align}

\newgeometry{top=5mm, bottom=5mm}
%your text here  
\begin{landscape}
با جاگذاری معادل‌سازی‌های ذکر شده برای رابطه‌ی
\ref{eq:field_zeroth_step}
جملات آن را بازنویسی می‌کنیم تا به این ترتیب مرحله‌ی اول روش بازگشتی به سرانجام برسد. برای مراحل بعدی نیز باید همین معادل‌سازی‌ها را تکرار کنیم. این روند باعث می‌شود تا 
E(t)
برحسب بسط جملاتی نوشته شود که حاصل اتفاقات زمان‌های گذشته هستند.

	\begin{align}
		E(t) &= \A{t-d} - g\int_{0}^{\infty} \N{t-d-u_1} \cdot \Pexp{1} E(t-d-u_1) du_1\\
		&= \A{t-d} - g\int_{0}^{\infty} \N{t-d-u_1} \cdot \Pexp{1} \cdot \big[ \A{t-2d-u_1} - g\int_{0}^{\infty} \N{t-d- u_1-u_2} \cdot \Pexp{2} E(t-2d-u_1-u_2) du_2 \big] du_1\\
		&= \A{t-d} - g\int_{0}^{\infty} \N{t-d-u_1} \cdot \Pexp{1} \cdot \A{t-2d-u_1} du_1\\ 
		&\hspace{1 cm}+ g^2 \int_{0}^{\infty} \N{t-d-u_1} \cdot \Pexp{1}\int_{0}^{\infty} \N{t-2d-u_1-u_2} \cdot \Pexp{2} E(t-2d-u_1-u_2) du_2 du_1\\
		&= \A{t-d} - g\int_{0}^{\infty} \N{t-d-u_1} \cdot \Pexp{1} \cdot \A{t-2d-u_1} du_1\\ 
		&\hspace{1 cm}+ g^2 \int_{0}^{\infty} \N{t-d-u_1} \cdot \Pexp{1}\int_{0}^{\infty} \N{t-2d-u_1-u_2} \cdot \Pexp{2} \A{t-3d-u_1-u_2} du_2 du_1\\
		&\hspace{1 cm}- g^3 \int_{0}^{\infty} \N{t-d-u_1} \cdot \Pexp{1}\int_{0}^{\infty} \N{t-2d-u_1-u_2} \cdot \Pexp{2} \int_{0}^{\infty} \N{t-3d-u_1-u_2-u_3} \cdot \Pexp{3} E(t-3d-u_1-u_2-u_3) du_3 du_2 du_1
	\end{align}
\end{landscape}

\restoregeometry 

حال اگر عمر این سامانه کراندار باشد؛ تعداد جملات بالا محدود می‌شوند. پس اگر سامانه پیش از یک زمانی کاملا خاموش بوده باشد$E = 0$؛ آنگاه می‌توان میدان کنونی را بر اساس جملات ضربی بین شدت جریان و تعداد نورون‌های تیزه زده پیدا کرد.\\
اگر چه به نظر نویسنده این راه‌حل بسیار درخشان است اما فعلا اطلاعات لازم برای ادامه‌ی آن را در اختیار نداریم. پس بهتر است آن را در گوشه‌ای نگه‌داریم تا در مراحل بعدی از آن استفاده کنیم.


\زیرقسمت{روش اختلال}
\زیرزیرقسمت{محاسبه‌ی حالت پایستار - تلاش شماره‌ی ۱}
به نمودار \ref{fig:sigma_rotational} دقت کنید. در زمانی که تعداد نورون‌ها بی‌نهایت باشد؛ در فاز ناهم‌گام انحراف معیار میدان صفر خواهد شد. این به این معنی است که جریان در زمان ثابت خواهد ماند. پس بگذارید با علم بر این موضوع یک جواب معادله‌ی \ref{eq:simple_network} را در حالت حدی میدان ثابت $E_0$ معرفی کنیم.\\
با فرض ثابت بودن میدان، اندازه‌ی آن را محاسبه می‌کنیم. سپس مجدد به معادلات برمی‌گردیم و می‌پرسیم که در صورت جمع با یک جمله‌ی اختلالی کوچک این انحراف رشد خواهد کرد یا خیر. به عبارت دیگر آیا این جواب جاذب است.\\
\begin{align}
	\begin{cases}
		E_0 = \frac{1}{N}\int_{- \infty}^{t - d} \int n_a(\pi,u) \cdot \big[ a - g E_0 \big] da \cdot \alpha^2 u\, e^{-\alpha u} du \\
		\frac{\partial n_a}{\partial t} = - \frac{\partial n_a}{\partial \theta} \cdot (a - g E_0 )
	\end{cases}
\end{align}

یک راه خوب برای پیشبرد سطر اول معادلات آن است که از دو طرف آهنگ تغییرشان با زمان را بپرسیم. از آنجا که سمت چپ معادله ثابت است؛ سمت راست هم باید جوابی مشابه را حکایت کند.\\
\begin{equation}
	0 = \frac{dE_0}{dt} = \frac{\alpha^2 (t-d) e^{-\alpha (t-d)}}{N} \cdot [ - gE_0 \cdot \int n_a(\pi,t-d) da + \int n_a(\pi,t-d)\cdot a\,da ]
\end{equation}
مشخص است که کدام جمله از جملات ضربی بالا صفر است. پس برای $E_0$ خواهیم داشت:
\begin{equation}
	E_0 = \frac{1}{g}\cdot \frac{\int n_a(\pi,t-d)\cdot a\,da}{\int n_a(\pi,t-d) da }
\end{equation}

حال برای ادامه‌ی فرآیند نیاز داریم تا عبارت حاکم بر 
$n_a(\pi,t-d)$
را بدست آوریم. اگر فرض کنیم که نورون با جریان خارجی a، در زمان صفر در فاز
$\theta_0$
حضور داشته؛ جواب پیشنهادی ما برای چگالی حضور نورون از جنس تابع دلتاست:
\begin{align}
	n_a(\theta,t) &= \delta(\theta - \theta_a(t)) \\
	&= \delta(\theta + \theta_0 - (a - g E_0)t + 2 \floor{K^{(t)}_a}\pi )\\
	&= \delta( \theta - (a - g E_0)t + 2 \floor{K^{(t)}_a} \pi + \theta_0  )\\
	\Rightarrow n_a(\pi,t) &= \delta(  (2\floor{K^{(t)}_a} + 1)\pi - (a - g E_0)t + \theta_0   )\\
\end{align}
که در این معادلات 
$K^{(t)}_a$
کسری است که تعداد دور هر نورون را از آغاز تا کنون روایت می‌کند و ما مجبور به عقب کشیدن 
$2\pi$
فاز کامل پس از تیزه زدن آن به تعداد 
$\floor{K^{(t)}_a}$
شده‌ایم.
\footnote{دقت کنیم که معادله‌ی ذکر شده برای نورون‌هایی درست است که 
	$(a - g E_0) > 0 $
}
قابل محاسبه است که عبارت کامل آن به صورت زیر است.
\begin{equation}
	K^{(t)}_a = \frac{(a - gE_0)t + \pi + \theta_0}{2\pi}
\end{equation}

برای محاسبه‌ی انتگرال‌هایی که شامل این دلتای دیراک هستند؛ لازم است تا صفر‌های آرگومان آن را محاسبه کنیم.
\begin{align}
	\big( 2 \floor{\frac{(a - gE_0)t + \pi + \theta_0}{2\pi}} + 1 \big)\pi - (a - g E_0)t + \theta_0 &= 0\\
	2\pi \times \bigg( \floor{\frac{(a - gE_0)t + \pi + \theta_0}{2\pi}}  - \frac{(a - gE_0)t + \pi + \theta_0}{2\pi} \bigg) &= 0\\
	2\pi \times \bigg( \floor{K^{(t)}_a} - K^{(t)}_a \bigg) &= 0  \label{eq:neuron_pattern_simple_model}
\end{align}
این رابطه کاملا یک تابع تناوبی را توصیف می‌کند. یک تابع مقطع که در مکانی که آرگومان آن صحیح می‌شود؛ مقدار صفر به خود می‌گیرد. پس روشن است که توقع داشته باشیم. تعداد صفرهای این معادله به اندازه‌ی تعداد تناوبی است که در هر زمان در بازه‌ی جریان‌های داده شده دارد.
\begin{align}
	\Delta K^{(t)}_a  &= 1\\
	\Delta K^{(t)}_a &= \frac{t}{2\pi}\Delta a\\
	\Delta a &= \frac{2\pi}{t}
\end{align}
این دوره‌ی تناوب با افزایش زمان کوچکتر می‌شود. اگر تعداد نورون‌ها را به صورتی ترمودینامیکی بزرگ بگیریم؛ آنگاه به ازای هر دوره‌ی تناوب یک نورون حتما هست که روی محور آستانه قرار گرفته است.\\
حال که دوره‌ی تناوب 
$\Delta a$
را بدست آوردیم؛ می‌دانیم که ریشه‌های رابطه‌ی 
\ref{eq:neuron_pattern_simple_model}
چه زمانی رخ می‌دهند. فرض کنیم که اولین صفر در جریانی مثل
$a_m$
رخ می‌دهد. توجه کنید حتما اندازه‌ی این جریان به گونه‌ای است که نورون را به صورت فعال نگه دارد. پس باید حتما
$(a_m - g E_0) > 0 $
باشد.
حال می‌توانیم انتگرال‌های مورد نظر خود را این چنین بسط دهیم.

\begin{align}
	\int n_a(\pi,t-d)a\,da &= \int \delta \bigg( 2\pi ( \floor{K^{(t)}_a} - K^{(t)}_a) \bigg) a\,da\\
	&= \frac{1}{2\pi} \cdot \sum_{K^{(t)}_a \in Z} a_i \\
	&= \frac{1}{2\pi} \cdot \sum^{M}_{m=0} a_m + m \cdot \Delta a\\
	&= \frac{M+1}{2\pi} \cdot ( \frac{a_m +a_{max} }{2} )\\
\end{align}
و از طرفی:
\begin{align}
	\int n_a(\pi,t-d)\,da &= \int \delta \bigg( 2\pi ( \floor{K^{(t)}_a} - K^{(t)}_a) \bigg) a\,da\\
	&= \frac{1}{2\pi} \cdot \sum_{K^{(t)}_a \in Z} 1 \\
	&= \frac{1}{2\pi} \cdot \sum^{M}_{m=0}  1\\
	&= \frac{M+1}{2\pi}
\end{align}
حال اگر به محاسبه‌ی میدان ثابت خود برگردیم و تکه‌های پازل را کنار هم بگذاریم؛ خواهیم داشت:
\begin{align}
	E_0 &= \frac{1}{g}\cdot \frac{\int n_a(\pi,t-d)\cdot a\,da}{\int n_a(\pi,t-d) da } \\
	&= \frac{1}{g}\cdot \frac{ \frac{M+1}{2\pi} \cdot ( \frac{a_m +a_{max} }{2} ) }{ \frac{M+1}{2\pi} } \\
	&= \frac{1}{g} ( \frac{a_m +a_{max} }{2} )
\end{align}
این میدان معادل است با جریان میانگین بین نورون‌هایی که آن‌ها را روشن خطاب کرده بودیم. این نتیجه صحیح نیست زیرا اگر میدان در میانه‌ی این جریان‌ها قرار گیرد؛ آنگاه نورون‌های با جریان پایین‌دست 
$a < \frac{1}{g} ( \frac{a_m +a_{max} }{2} ) $
را خاموش خواهد کرد و اصلا روشن نخواهند ماند. پس این راه‌حل نیز دارای مشکل است و تا مشخص شدن نقص آن، آن را کنار می‌گذاریم.

\زیرزیرقسمت{محاسبه‌ی حالت پایستار - تلاش شماره‌ی ۲}
در این روش فرض می‌کنیم که برای هر جریان تصادفی اولیه، نورون‌های زیادی را به اختیار گرفته‌ایم. در حالت پایا  ، در یک حالت خاص تغییری در چگالی جمعیت مشاهده نمی‌شود پس در معادله‌ی
\ref{eq:simple_network}
خواهیم داشت:
\begin{equation}
	\frac{\partial n_a}{\partial t} = 0
\end{equation}
همچنین در حالت پایا که در واقع از نگاه ما حالت ناهم‌گام است؛ جریان بین نورون‌ها - که کمیتی بزرگ مقیاس است -  در زمان تغییری نمی‌کند. پس به این ترتیب:

\begin{align}
	\begin{cases}
		\frac{\partial n_a}{\partial t} = - \frac{\partial J_{a}(t)}{\partial \theta} = 0\\
		J_{a}(\theta, t) = n_a(\theta,t) \cdot [ a - g E ]\\
	\end{cases}
	\Rightarrow J_{a}(\theta, t) = J_{a}(t)\\
	\Rightarrow n_{a}(\theta, t) = n_{a}\\
\end{align}

پس توزیع جمعیت نورون‌‌ها مستقل از زمان و حالت آن‌ها خواهد شد. اگر توزیع را در ابتدا یکنواخت میان جریان‌های مختلف توزیع کرده باشیم؛ برای همه‌ی زمان‌ها و حالت‌ها داریم:
\begin{equation}
	n = \frac{N}{2 \pi (a_{Max} - a_{min}) }
\end{equation}


برای جریان بین نورون‌ها هم خواهیم داشت:
\begin{align}
	E &= \frac{1}{N} \int_{- \infty}^{t - d} \int n \cdot \big[ a - g E \big] da \cdot \alpha^2 u\, e^{-\alpha u} du\\
	&=  \int \frac{n}{N} \cdot \big[ a - g E \big] da \label{eq:e_sum_simple}
\end{align}
دقت کنیم که انتگرال رابطه‌ی \ref{eq:e_sum_simple} روی  نورون‌هایی است که مستعد تیزه زدن هستند.
\footnote{
	$(a - g E) > 0 $
}

اولین جریانی که نورون را مستعد تیزه زدن می‌کند $a_*$ نام‌گذاری می‌کنیم. وقتی جریان مهاری حاصل از تیزه زدن‌ها کوچک است؛ همه‌ی نورون‌ها فعال هستند و در نتیجه‌ 
$a_* = a_{min}$
می‌شود. اما در حالتی که جریان مهاری زیاد می‌شود؛ این مقدار از کمترین جریان تصادفی اولیه سامانه بزرگتر می‌شود. محاسبات را ادامه می‌دهیم:
\begin{align}
	E &=  \int \frac{n}{N} \cdot \big[ a - g E \big] da \\
	&= \frac{n}{N} \cdot \big[ \frac{a^2_{Max} - {a_*}^2}{2} - g E (a_{Max} - a_*) \big] \\
	\Rightarrow E &= n \cdot \big[ \frac{a^2_{Max} - {a_*}^2}{2}  \big] / \big[ N + g n (a_{Max} - a_*)\big]
	\label{}
\end{align}
شاید بنظر این یک معادله‌ی درجه یک ساده باشد که میدان را گزارش می‌کند اما در واقع خود $a^*$ هم به میدان وابسته است و باید وابستگی آن را لحاظ کنیم. به تقریب:
$a^{*} = gE$
با اضافه کردن این معادله و حل معمول یک معادله‌ی درجه‌ی دو برای میدان صراحتا خواهیم داشت:
\begin{align}
	E =  \big( \frac{a_{Max}}{g} + \frac{N}{n g^2}  \big) \pm \big[ \big( \frac{N}{n g^2} + \frac{a_{Max}}{g} \big)^2 - \frac{{a_{Max}}^2}{g^2} \big]^{\frac{1}{2}}
\end{align}
نتیجه می‌دهد که $a_{*}$ هم باید به صورت زیر باشد:
\begin{align}
	a_{*} &=  \big( a_{Max} + \frac{N}{n g} \big) \pm \big[ \big( \frac{N}{n g} + a_{Max} \big)^2 - {a_{Max}}^2 \big]^{\frac{1}{2}}\\
	&=  \big( a_{Max} + \frac{N}{n g} \big) \pm \big[  \frac{N^2}{n^2 g^2} + \frac{2a_{Max}N}{ng}  \big]^{\frac{1}{2}}
\end{align}
اجازه بدهید علامت مثبت را کنار بگذاریم زیرا مقدار $a_{*}$ را خارج بازه‌ی جریان‌های سامانه گزارش می‌کند. پس هم برای میدان و هم جریان $a_{*}$ خواهیم داشت:
\begin{align}
	\begin{cases}
		a_{*} =  \big( a_{Max} + \frac{N}{n g} \big) - \big[  \frac{N^2}{n^2 g^2} + \frac{2a_{Max}}{ng}  \big]^{\frac{1}{2}}\\
		E =  \big( \frac{a_{Max}}{g} + \frac{N}{n g^2} \big) - \big[ \frac{N^2}{n^2 g^4} + \frac{2Na_{Max}}{ng^3}  \big]^{\frac{1}{2}}\\
	\end{cases}
\end{align}

حال اگر نتایج بدست آمده را با داده‌های شبیه‌سازی تطبیق دهیم؛ خواهیم دید که تطابق خوبی با یک دیگر دارند.

\begin{figure}
	\centering
	\begin{subfigure}[b]{0.7\textwidth}
		\centering
		\includegraphics[width=\textwidth]{../scripts/all_neurons_model_in_one_place/Non_repulsive_rotational_ensembles/N10000_T100_I9.5_13.5_v1.0/field_zeroth_order_asterix_g_0.1_130.png}
		\caption{نسخه‌ای که کمینه‌ی جریان را از حل محاسبات درنظر می‌گیرد}
		\label{fig:e_zeroth_not_all_gifted}
	\end{subfigure}
	\hfill
	\begin{subfigure}[b]{0.7\textwidth}
		\centering
		\includegraphics[width=\textwidth]{../scripts/all_neurons_model_in_one_place/Non_repulsive_rotational_ensembles/N10000_T100_I9.5_13.5_v1.0/field_zeroth_order_simple_g_0.1_130.png}
		\caption{نسخه‌ای که همه‌ی نورون‌ها را فعال تصور می‌کند.}
		\label{fig:e_zeroth_all_gifted}
	\end{subfigure}
	\hfill
	\begin{subfigure}[b]{0.7\textwidth}
		\centering
		\includegraphics[width=\textwidth]{../scripts/all_neurons_model_in_one_place/Non_repulsive_rotational_ensembles/N10000_T100_I9.5_13.5_v1.0/field_zeroth_order_developed_g_0.1_130.png}
		\caption{نسخه ساخته شده از اتصال هر دو حالت}
		\label{fig:e_zeroth_developed}
	\end{subfigure}
	\caption{تطابق جریان پایای بدست آمده از حل عددی و تحلیلی}
	\label{fig:three graphs}
\end{figure}

در ضریب تاثیرهای بسیار بزرگ داریم:
\begin{align}
	E \cong & \frac{a_{Max}}{g} + \frac{N}{ng^2} - (\frac{2N a_{Max}}{n g^3})^{\frac{1}{2}} \big[ 1 + \frac{N}{2nga_{Max}} \big]^{\frac{1}{2}}\\
	=& \frac{a_{Max}}{g} + \frac{N}{ng^2} - (\frac{2 N a_{Max}}{n g^3})^{\frac{1}{2}} \big[ 1 + \frac{N}{4nga_{Max}} \big]\\
	=& \frac{a_{Max}}{g} - (\frac{2 N a_{Max}}{n g^3})^{\frac{1}{2}} + \frac{N}{ng^2}  - (\frac{N}{2n})^{\frac{3}{2}}  \cdot \frac{1}{ {a^{\frac{1}{2}}_{Max} g^{\frac{5}{2}}}}
\end{align}


\زیرزیرقسمت{اختلال در میدان پایستار - تلاش شماره‌ی ۱}
همان طور که مشخص است؛ حل دقیق میدان بسیار کار دشواری است اما می‌توان از طریق ترفندهای اختلالی به جواب آن نزدیک شد. یکی از روش‌های معمول حل زنجیری و تودرتوی دستگاه معادلات است. \\
به این ترتیب که ابتدا از معادله  پاسخ حالت پایا (مرتبه‌ی صفرم) را در معادله‌ی پخش جاگذاری می‌کنیم تا توزیع آماری وابسته به زمان نورون‌ها بدست آید. سپس مجددا از توزیع بدست آمده؛ میدان مرتبه‌ی اول را که وابسته به زمان است؛ محاسبه می‌کنیم.\\
از آنجا که توزیع سامانه‌ رفتاری دوره‌ای به طول $2\pi$ دارد؛ می‌توانیم آن را به صورت زیر بسط دهیم:
\begin{align}
	\rho(\theta, a, t) = \rho_0 + \sum_k A_k(t) e^{ik\theta}, \quad k \in \mathcal{Z}
\end{align}

\begin{align}
	\frac{\partial \rho}{\partial t} &= \sum \dot A_k e^{ik\theta}\\
	\frac{\partial \rho}{\partial \theta} &= \sum A_k \cdot ik \cdot e^{ik\theta}\\
\end{align}
حال آن را در معادله‌ی پخش قرار می‌دهیم تا بتوانیم معادله‌ی حاکم بر ضرایب را محاسبه کنیم.
\begin{align}
	\sum \dot A_k e^{ik\theta} &= - \sum A_k \cdot ik(a - gE(t)) \cdot e^{ik\theta}\\
	\Rightarrow \dot A_k &= - A_k \cdot ik(a - gE(t))
\end{align}
در تقریب مرتبه‌ی اول برای توزیع داریم:
\begin{align}
	\dot A_k &= - A_k \cdot ik(a - g E_0)\\
	\Rightarrow A_k(t) &= A_k(0) e^{- ik(a-gE_0)t}\\
	\Rightarrow \rho(\theta, a, t) &= \rho_0 + \sum_k A_k(0) e^{ik\theta - ik(a - gE_0)t}
\end{align}
پس برای نورون‌های روی آستانه خواهیم داشت:
\begin{align}
	\rho(\pi, a, t) = \rho_0 + \sum_k A_k(0) e^{ik\pi - ik(a - gE_0)t}
\end{align}
حال از نتیجه‌ی بدست آمده استفاده می‌کنیم و همان طور که اشاره شد به محاسبه‌ی مرتبه‌ی بعدی میدان می‌رویم:
\begin{align}
	E(t) &= \int \int_{0}^{\infty} \rho(\pi, a, t-d-v)\cdot \dot \theta \cdot \alpha^2 ve^{-\alpha v} dv da \label{eq:perturbed_field}\\
	&= E_0\\
	&+ \int \int_{0}^{\infty} \sum_k A_k(0) e^{ik\pi - ik(a - gE_0)(t-d - v)} \cdot (a - gE_0) \alpha^2 ve^{-\alpha v} dv da \\
	&= E_0 + \sum_k \int \int_{0}^{\infty} A_k(0) e^{ik\pi - ik(a - gE_0)(t-d - v)} \cdot (a - gE_0) \alpha^2 ve^{-\alpha v} dv da
\end{align}
اجازه بدهید سهم مدهای متفاوت از میدان را به صورت جداگانه محاسبه کنیم و سپس مجددا در کنار یکدیگر قرار دهیم.
\begin{align}
	E_{k,a}(t) = -\alpha^2 A_k(0)(a - gE_0) e^{ik\pi}\int^{\infty}_{0} v e^{-[\alpha - ik(a - gE_0)]v-ik(a - gE_0)(t-d)} dv
\end{align}
با با تغییر متغیر 
$\beta \equiv \alpha - ik(a-gE_0)$
محاسبات را ادامه می‌دهیم:
\begin{align}
	&= \alpha^2 A_k(0) (a - gE_0) e^{ik\pi} e^{-ik(a-gE_0)(t-d)} \cdot \int^{\infty}_0 v e^{-\beta v}dv \\
	&= \alpha^2 A_k(0) (a - gE_0) e^{ik\pi} e^{-ik(a-gE_0)(t-d)} \cdot \frac{1}{\beta^2} \\
	&= A_k(0) (a - gE_0) e^{ik[\pi -(a-gE_0)(t-d)]} \cdot (\frac{\alpha}{\alpha-ik(a-gE_0)})^2
\end{align}
حال قدم به قدم به محاسبات پیشین خود برمی‌گردیم. ابتدا می‌پرسیم میدان همه‌ی نورون‌های با مد یکسان چه جریانی را تولید می‌کنند.
\begin{align}
	E_k(t) &= \int E_{k,a} da\\
	&= \int A_k(0) (a - gE_0) e^{ik[\pi -(a-gE_0)(t-d)]} (\frac{\alpha}{\alpha-ik(a-gE_0)})^2 da
\end{align}
با تغییر متغیر
$h \equiv a - gE_0$
تلاش می‌کنیم انتگرال را ادامه دهیم.
\begin{align}
	E_k(t) = A_k(0)e^{ik\pi} \int^{a_M - gE_0}_{0} h e^{-ikh(t-d)} (\frac{1}{1 - ikh/\alpha})^2 dh
\end{align}
نرم‌افزارهای محاسباتی همچون ابزار ولفرم به ما امکان می‌دهد تا پاسخ آن را به صورت زیر بیان کنیم:
\begin{align}
	E_k(t) = -A_k(0)\frac{\alpha^2}{k^2} e^{ik\pi} &\bigg[ \frac{ e^{-i(\xi_{(h)} + k(t-d)h)} }{\sqrt{1 + h^2 k^2/\alpha^2}} \\
	&+ e^{-\alpha(t-d)}(\alpha(t-d) + 1) Ei[(\alpha-ikh)(t-d)] \, \bigg]	\Biggr|_{0}^{a_M -gE_0}
\end{align}
به صورتی که 
$e^{-i\xi_{(h)}} = \frac{1 + ikh/\alpha}{\sqrt{1 + h^2 k^2/\alpha^2}}$
است و 
$Ei$
همان تابع انتگرال نمایی است که به صورت 
$Ei[z] = - \int^{+\infty}_{-z} e^{-t}/t dt$
نوشته می‌شود.
\begin{align}
	E_k(t) =& - A_k(0) \frac{\alpha^2}{k^2} e^{ik\pi} \bigg[ \frac{ e^{-i(\xi_{(a_M -gE_0)} + k(t-d)(a_M -gE_0))} }{\sqrt{1 + (a_M -gE_0)^2 k^2/\alpha^2}} \\
	&+ e^{-\alpha(t-d)}(\alpha(t-d) + 1) Ei[(\alpha-ik(a_M -gE_0))(t-d)] \\
	&- e^{-ik(t-d)(a_M -gE_0)} \\
	&- e^{-\alpha(t-d)}(\alpha(t-d) + 1) Ei[\alpha(t-d)] \bigg]\\
	=&- A_k(0) \frac{\alpha^2}{k^2} e^{ik\pi} \bigg[ e^{-ik(t-d)(a_M -gE_0)} \bigg( \frac{ e^{-i(\xi_{(a_M -gE_0)})} }{ \sqrt{1 + (a_M -gE_0)^2 k^2/\alpha^2} } + 1\bigg) \\
	&+ e^{-\alpha(t-d)}(\alpha(t-d) + 1)\bigg( Ei[(\alpha-ik(a_M -gE_0))(t-d)] - Ei[\alpha(t-d)] \bigg) \bigg]
\end{align}
پس یک جمله‌ی نوسانی دارد و جمله‌ای که شامل تکینگی است.\\
خبر خوب یا بد این است که این راه هم دارای ایراد است. زیرا در محاسبه‌ی مرتبه‌ی اول میدان اشتباهی رخداده است - رابطه‌ی \ref{eq:perturbed_field}.
در این رابطه باید ضرب مرتبه‌ی صفرم چگالی در مرتبه‌ی اول میدان جا مانده است. یعنی باید می‌نوشتیم:
\begin{align}
		E(t) &= \int \int_{0}^{\infty} (\rho_0 + \rho_1)(a - g(E_0 + E_1) ) \cdot \alpha^2 ve^{-\alpha v} dv da \\
		&= \int \int_{0}^{\infty} [\rho_0(a - gE_0) + \rho_1 (a - gE_0)] \cdot \alpha^2 ve^{-\alpha v} dv da \\
		&+ \int \int_{0}^{\infty} -g E_1 \rho_1 \cdot \alpha^2 ve^{-\alpha v} dv da \\
		&+ \int \int_{0}^{\infty} -g E_1 \rho_0 \cdot \alpha^2 ve^{-\alpha v} dv da \\
\end{align}
در واقع جمله‌ی آخر رابطه‌ی بالا جامانده بود و باعث می‌شود بخشی از جواب در پشت آن پنهان بماند. البته با در نظر گرفتن آن جمله پیچیدگی اصلی مسئله دوباره به صفحه‌ی بازی برمی‌گردد.
 
 
 
 
 
 
 
 


