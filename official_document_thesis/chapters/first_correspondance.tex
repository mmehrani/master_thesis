\قسمت*{شروع داستان}
در ابتدای تابستان پیش از شروع دوره‌ی ارشد؛ از برنامه‌نویسی رایانه‌های کوانتومی خبردار شدم. اگر این پردازنده‌ها از پایه، طریق محاسبه‌ی متفاوتی دارند؛ پس باید برنامه‌نویسی آن‌ها نیز تفاوت کند. این یعنی در دهه‌های پیشرو دیگر خبری از برنامه‌نویسی کلاسیکی نخواهد بود. این رایانه‌ها ساخته شده‌اند تا محاسبات پیچیده را با الگوریتم‌هایی حل کنند تا مرتبه‌ی زمان لازم برای حل آن‌ها به طریق خیره کننده‌ای کاسته شود.\\
پس این ایده در ذهن من متبلور شد که آیا این فناوری می‌تواند در حل مسائلی که آن‌ها را پیچیده می‌نامیم هم کاربرد داشته باشد یا خیر.  آیا آینده‌ی این دو گرایش به یک دیگر گره خواهد خورد؟ شروع کردم به پرسش از اساتید دانشکده و کسانی که در این دو گرایش فعال و متخصص بودند.\\
جملات زیر شرح متخصری از نظرات چندتن از اساتید متخصص است. این جملات نقل به مضمون هستند.
\begin{enumerate}
	\item 
	دکتر صادق رئیسی: امروز در مورد معماری رایانه‌های کوانتومی تحقیق می‌کنیم و هنوز تا کاربرد مستقیم این رایانه‌ها فاصله داریم. پس از بلوغ این رایانه‌ها علی‌الاصول متخصصان این حوزه برنامه‌نویسی آن را به گرایش‌هایی همچون سامانه‌های پیچیده خواهند آموخت و علاقه‌مندی به برنامه‌نویسی کوانتومی شرط کافی برای ورود به این حوزه نیست.
	\item
	دکتر سامان ابوالفتح بیگی: معماری این رایانه‌ها به گونه در حال توسعه است که آن‌ها را دوگانه‌سوز \footnote{هیبرید} پیش ببرد. پس با اضافه شدن معماری کوانتومی لزوما شیوه‌ی کار با آن‌ها متفاوت از رایانه‌های کلاسیکی نخواهد شد و این تفاوت تنها در لایه‌های محاسباتی خواهد ماند.
	\item 
	دکتر وحیدکریمی‌پور: مطمئنا شیوه‌ی حل مسائل در عصر رایانه‌های کوانتومی تغییر خواهد کرد. اولین تفاوت مهم بین منطق این دو رایانه‌ها این است که رایانه‌های کلاسیکی «گزاره‌های شرطی» را می‌فهمند و در سمت دیگر اصلا حالت صحیح و غلط اصلا مطلق نیست وو برهم‌نهی از این دو حالت وجود دارد. پس اگر کسی می‌خواهد اصول کار با این رایانه‌های را بیاموزد باید از معماری آن‌ها نیز اطلاع داشته باشد.

\end{enumerate}



\قسمت*{شروع مکاتبات}
پس از تمام این صحبت‌ها به این نتیجه رسیدم؛ زمانی که در آن زندگی می‌کنم هنوز فاصله‌ی زیادی با آینده‌ی ایده‌آل من دارد؛ پس تصمیم گرفتم همچنان به علاقه‌ی اول خود یعنی «سامانه‌های پیچیده» برگردم و آن را فعلا به شکل کلاسیکی دنبال کنم. از آن‌جا که رابطه‌ی بسیار خوبی با استاد سامان مقیمی داشتم و همچنین از متخصصان نامی این حوزه هستند؛ تصمیم گرفتم تا درخواست خودم را با یک نامه الکترونیکی به ایشان تقدیم کنم - پیوست
\ref{appendix:letter_13990828}
.


\قسمت*{نحوه‌ی ارجاع به مکاتبات}
برای خواندنی کردن هر چه بیشتر این پایان‌نامه، تلاش کردم تا گوشه‌ای از مکاتباتی که با استادم داشتم را گردآوری کنم. فکر می‌کنم اشاره به آن‌چه که  در پس پرده‌ی این تلاش‌ها اتفاق افتاد می‌تواند در درک هر چه بیشتر آن کمک کند و چهره‌ای واقعی‌تر از یک سخن علمی را به تصویر بکشد. پس برای تحقق این هدف پاره‌ای از متن رایانامه‌هایی را که مبادله شد؛ در میان متن آورده‌ام. این رایانامه‌ها به با صفحاتی با رنگ متمایز به مانند صفحه‌ی قبل مشخص شده‌اند.
\footnote{
به درخواست داوران این پایان‌نامه، در مورخ ۳۰ شهریور ۱۴۰۱، این نامه‌ها به پیوست 
\ref{chap:letters}
منتقل شدند.
}
