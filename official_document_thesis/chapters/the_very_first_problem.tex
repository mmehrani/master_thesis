\قسمت{شرح مسئله}
\زیرقسمت{علوم اعصاب نیازمند آرامش}
مطالعه فعالیت شبکه‌های عصبی برای تحقیق و بررسی کارکردهای مغز اهمیت زیادی دارد. همه بر این باوریم که مغز محمل اندیشه و تفکر است. ما کنجکاو هستیم که چگونه همکاری بین نورون‌های آن باعث می‌شود تا حافظه، کشف و پردازش صورت گیرد. هر کدام از نورون‌های مغز می‌تواند در حالت فعال [روشن] یا غیرفعال [خاموش] قرار گیرد. هم اکنون شواهدی وجود دارد که بیماری صرع در زمان‌هایی رخ می‌دهند که الگوی خاموش و روشن شدن نورون‌های آن باهم \textbf{«هم‌گامی»} دارند. هم‌گامی به این معناست که جمعیت بزرگی از نورون‌ها هم باهم خاموش و روشن می‌شوند و یک الگوی تکرار شونده‌ای را دنبال می‌کنند. تو گویی که باهم هم‌آهنگ یا هم‌گام شده‌اند.\\

بی‌تردید دستیابی به تمام جزییات مغز برای ما میسّر نیست و به آن به عنوان یک \textbf{«جعبه‌ی سیاه»} نگاه می‌کنیم که مدت‌هاست به دنبال ارائه مدلی هستیم که رابطه‌ی بین ورودی‌ها و خروجی‌های ثبت شده را بازتولید کند.  کاری که در این پژوهش انجام خواهیم داد تلاشی است برای پیشنهاد دادن یک مدل برای این جعبه‌ی سیاه که رفتار نسبتا مشابهی را میان ورودی و خروجی‌های این جعبه سیاه شبیه‌سازی می‌کند. همچنین به کمک بررسی دقیق‌تر این مدل تلاش می‌کنیم تا نقطه‌ی تقریبی گذرفاز سامانه را از فاز ناهم‌گام به هم‌گام پیدا کنیم.

\زیرقسمت{مسئله‌ات را به من بده!}
مدل‌های زیادی برای شبکه‌های عصبی ارائه شده است که توانایی تولید هم‌گامی نورون‌ها را در آن‌ها می‌توانیم جستجو کنیم. یکی از این مدل‌ها که در تمام فصول شبیه‌سازی از آغاز تا کنون از آن بهره برده شده است؛ مدل انباشت و شلیک است\cite{brunel2007quantitative}. 

جستار خود را این گونه پیش می‌بریم:
\begin{enumerate}[(a)]
	\item
	ابتدا مدل پیشنهادی نویسندگان \cite{PhysRevLett.105.158104} را بازتولید می‌کنیم و تلاش می‌کنیم تا نتایج آن‌ها را دوباره بدست آوریم. بنظر می‌آید تمام کمیت‌های مهم را اندازه‌گیری نکرده‌اند. پس جوانب دیگر مسئله را هم بررسی خواهیم کرد. [به فصل 
	\nameref{chap:integrate_and_fire}
	نگاه کنید.
	]
	\item 
	«آیا این رفتار همگامی برای مدل‌های نورونی دیگر نیز اتفاق خواهد افتاد؟ یا فقط با این مدل این رفتار را مشاهده خواهیم کرد؟» پس مدل نورونی خود را عوض می‌کنیم تا پاسخ این پرسش را دریابیم. مدل پیشنهادی و جایگزین ما «چرخنده» است. [به فصل 
	\nameref{chap:rotational}
	نگاه کنید.
	]
	\item 
	تلاش کنیم تا مکان تغییر فاز به همگامی را با مدل‌های تحلیلی بدست آوریم. این کار برای مدل انباشت‌وشلیک در این مقاله 
	\cite{brunel2000dynamics}
	انجام شده است اما آیا می‌توانیم برای مدل‌های نورونی دیگر نیز آن را محاسبه کنیم؟ [به فصل 
	\nameref{chap:analytics}
	نگاه کنید.
	]
\end{enumerate}
