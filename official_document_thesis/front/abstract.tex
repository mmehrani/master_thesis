
% -------------------------------------------------------
%  Abstract
% -------------------------------------------------------


\pagestyle{empty}

\شروع{وسط‌چین}
\مهم{چکیده}
\پایان{وسط‌چین}
پس از صده‌ها که انسان همیشه ناظر پدیده‌های در طبیعت بود؛ ظهور چارچوب جدیدی از مسائل با عنوان «علوم اعصاب» وجودیت خود را نیز به چالش کشید. از زمان تولد این علم تا کنون دانشمندان زیادی پدیده‌هایی همچون تفکر، خوب و بیماری‌های حافظه را با کاکرد سلول‌های عصبی توصیف کرده‌اند. یکی از مهم‌ترین شاخه‌های این علم بررسی مسئله‌ی «هم‌گامی» است. زمانی که کارکرد این سلول‌های عصبی هم‌زمانی و یا هماهنگی با یکدیگر پیدا می‌کند. این که چطور این سلول‌ها با در کنار هم قرار گرفتند از یک نا‌هم‌گامی به هم‌گامی تغییر فاز می‌دهند؛ محل بحث پژوهشگران زیادی است. مدلسازی‌های بسیاری برای توجیه این گذار ارائه شده است. ما در جستار خود به مشاهده‌ی پدیده‌ی هم‌گامی در شبکه‌های انباشت‌وشلیک و چرخنده می‌پردازیم و البته نشان می‌دهیم که این گذار فاز تا حدودی مستقل از نوع مدل نورونی است و در مدل‌های ساده‌تری نیز رخ می‌دهد. همچنین نشان می‌دهیم که این هم‌گامی از جنس قفل‌شدگی سرعت‌هاست. به این معنی که در فاز هم‌گامی همه‌ی نورون‌ها به سمت آستانه‌ی تیزه هم‌سو و هم‌سرعت سوق پیدا می‌کنند. پدیده‌ی هم‌گامی برای ما بسیار جذاب و با اهمیت زیرا دیده‌شده است که الگوی مغزی بیماران صرعی بسیار شبیه به الگوی هم‌گام است. گویا وقتی بیمار به حالت تشنج می رود؛ بخش بزرگی از نورون‌های مغزی او تغییر فازی به سمت حالت هم‌گام داده‌اند.
\\
\پرش‌بلند
\بدون‌تورفتگی \مهم{کلیدواژه‌ها}: 
سلول عصبی، هم‌گامی، صرع، شبیه‌سازی
\صفحه‌جدید