\فصل{سرآغاز}
به یاری خداوند برگی دیگر از داستان زندگی من در دانشگاه شریف ورق خورد. پس از انتظار بسیار و دوران طولانی کارشناسی و ارشد، بالاخره زمانی رسید که اثری کوچک با نام خودم در دنیای بی‌انتهای علم برجای بگذارم. برای تمام فیزیک‌پیشگان و متخصصان علمی کاملا واضح و ملموس است که این مسیر همواره پر از رنج، سختی و لذت است. حال که به این نقطه رسیده‌ام تا نامه‌ای را در پایان به جامعه علمی تقدیم کنم؛ دریغ است که از سختی‌های مسیر ننویسم. از راهی که هر متخصص و پیشگام علمی باید تجربه کند.\\

به همین خاطر می‌خواهم پایان‌نامه‌ام را به شیوه‌ی داستانی نگارش کنم. این نامه، نه تنها شامل نتایج پژوهش استاد ارجمندم «سامان مقیمی عراقی»  و من است بلکه شامل راه‌های بن‌بستی نیز هست که در این دوسال به آن برخوردیم. بعضا شامل حدس‌هایی است که در ابتدا زدیم و صحیح نبودند. شامل مکاتباتی است که بین من و استادم صورت می‌گرفت.\\

مایل هستم تا از آن بن‌بست‌ها هم بنویسم هر چند که نگارش آن‌ها نیازمند متحمل شدن سختی دیگری است. پس این نامه را نه تنها ارزشمند می‌دانم به خاطر مطالب علمی آن؛ بلکه یک روایت و رمانی است از مسیری که یک دانشجوی ارشد باید در دوران تحصیل خود تجربه کند. هم شامل مقصد است و هم از راه روایت می‌کند.